
\documentclass{article}

% Packages for formatting
\usepackage[utf8]{inputenc}
\usepackage[T1]{fontenc}
\usepackage{geometry}
\usepackage{titling}
\usepackage{enumitem}
\usepackage{amsmath}
\usepackage{amsfonts}
\usepackage{amssymb}
\usepackage{graphicx}
\usepackage{hyperref}
\usepackage{fancyhdr}
\usepackage{lipsum} % For generating dummy text, you can remove this in your final document
\usepackage{xcolor}
\usepackage{tcolorbox}

% Define page geometry
\geometry{a4paper, margin=1in}

% Define title
\title{Study Material Template}
\author{Your Name}
\date{\today}

% Define header and footer
\pagestyle{fancy}
\fancyhf{}
\rhead{\thepage}
\lhead{\theauthor}
\chead{\thetitle}
\renewcommand{\headrulewidth}{1pt}
\renewcommand{\footrulewidth}{0pt}

% Define colors
\definecolor{controllable}{RGB}{50, 132, 191}
\definecolor{observable}{RGB}{65, 168, 95}
\definecolor{observalternative}{RGB}{191, 97, 106}

% Define tcolorbox styles
\tcbset{
    commonstyle/.style={
        boxrule=0.5mm,
        colback=white,
        arc=2mm,
        fonttitle=\bfseries
    },
    controllability/.style={
        commonstyle,
        colframe=controllable
    },
    observability/.style={
        commonstyle,
        colframe=observable
    },
    alternative/.style={
        commonstyle,
        colframe=observalternative
    }
}

\begin{document}

% Title
\maketitle

% Table of contents
\tableofcontents
\newpage

% Start of content

\section{Introduction}
% Your introduction here

\section{Inverse kinematics}

% \section{Dynamics}
\section{Dynamics}



\subsection{Recall}

Let's recall the main formulas in dynamics


\begin{minipage}{0.45\textwidth}
    \begin{equation}
    \textbf{Linear Momentum: } p = mv
    \end{equation}
    where $m$ is the mass,\\
    $v$ is linear velocity.
    \end{minipage}
    \hfill
    \begin{minipage}{0.45\textwidth}
    \begin{equation}
    \textbf{Angular Momentum: } L = I\omega
    \end{equation}
    where $I$ is the moment of inertia,\\
    $\omega$ is angular velocity.
\end{minipage}


Euler's laws
% http://emweb.unl.edu/NEGAHBAN/EM373/note21/note21.htm

\subsection{Generalized forces. Generalized coordinates}


\subsection{Intro to dynamics}

Dynamics in Robotics stablishes the relation between generalized forces on the robot {$\tau(t)$} and 
robot motion {$q(t)$, $\dot q(t)$, $\ddot q(t)$}. 

We can speak about torques as torques at the joints, external forces or torques applied from the environment. 

When we speak about robot motion, we usually speak about joint trajectory or Cartesian trajcetory (of the end-effector). 

The dynamic model of a rigid body is described by the following equation:

\[M(q) \ddot q + C(q, \dot q) \dot q + g(q) = \tau\]
So, when we do the dynamics model, we search for: 

\begin{enumerate}
    \item M(q) - inertia of mass matrix
    \item C($q, \dot q$) - corioulis and centrifugal forces
    \item g(q) - gravity forces
    \item $\tau$ - generalized forces
\end{enumerate}

The task of \textbf{Forward dynamics} is  obtaining the robot motion $\ddot q$ based on applied forces, torques $\tau$. 

The task of \textbf{Inverse dynamics} is obtaining the forces or torques $\tau$ needed to achieve a desired motion $\ddot q$.  

% A scheme 

 % \textbf{Angular moment}

\subsection{Approaches for dynamic modeling}
Two core approaches for solving dynamics exist: Newton-Euler (force-torque based) and Euler-Lagrange (energy based). 

\subsubsection{Euler-Lagrange}


Euler-Lagrange approach is based on the system's energy

\begin{itemize}
    \item Kinetic energy
    \item Potential energy
\end{itemize}

\textbf{Lagrangian}
\[L(q, \dot q) = T(q, \dot q) - U(q) \text{, where $q$ are \textit{generalized} forces}\]

Euler-Lagrange equation:
\[\]


Intuition behind the Lagrangian. 

\subsection{Center of mass}

\begin{enumerate}
    \item The body moves so as if the mass is concentrated in the CoM. 
    \item All the applied forces are applied to the CoM.
\end{enumerate}


Center of mass for a rigid body:
\[p_c = \dfrac{1}{m} \int_m { \widetilde{p}  dm} = \dfrac{1}{m} \int_m {\widetilde{p} \rho dV}, \]

$\widetilde{p}$ is position of some point of the rigid body.

\subsubsection{Tensor of inertia}

Let's recall the formula:

\[A \times (B \times C) = (A \cdot C)B - (A \cdot B)\]

It is always symmetric.
% https://ocw.mit.edu/courses/16-07-dynamics-fall-2009/dd277ec654440f4c2b5b07d6c286c3fd_MIT16_07F09_Lec26.pdf

% http://www.kwon3d.com/theory/moi/iten.html



\subsection{Angular momentum}

- angular momentum in frame {0}
\[{}^0h = {}^0I  {}^0w\]

- angular momentum in frame {B}
\[{}^Bh = {}^BI  {}^Bw\]

Change from reference frame {B} to {0}


\subsection{Kinetics energy of a rigid body}
\subsection{Potential energy of a rigid body}

\subsection{Generalized forces. Generalized coordinates}

\end{document}