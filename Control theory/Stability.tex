
\section{Stability}


\subsection{Critical Point (Node)}

\begin{tcolorbox}[colback=green!10,colframe=green!50!black,title=\textbf{Critical Point (Node)}]
Consider the following LTI:
\[
\dot{x} = f(\mathbf{x}, t)
\]
\(x_0\) is called a \textbf{Node}, or \textbf{Critical Point}, if \(f(x_0) = 0\).
\end{tcolorbox}

\subsection{Stability}
\begin{tcolorbox}
A system is \textbf{stable} if:
\[
\|x(0) - x_0 \| < \delta \quad \| x(t) -x_0\| < \epsilon
\]
\end{tcolorbox}

We can think of it as: if the starting point is in the $\delta$-neighborhood of the node $x_0$, 
the rest of the trajectory $x(t)$ is in the $\epsilon$-neighborhood of the node.

Or, the solutions starting from the $\delta$-sized ball do not diverge. 

\subsection*{Asymptotic Stability}

\begin{tcolorbox}
A system is \textbf{asymptotically stable} if:
\[
\|x(0) - x_0 \| < \delta \to \lim_{t \to \infty} x(t) = x_0 
\]
\end{tcolorbox}

For any initial point that lies in the \(\delta\)-sized ball, the trajectory will asymptotically approach the node (point $x_0$).
Or, the solutions starting from the \(\delta\)-sized ball, converge to the node. 

Marginal Stability: Specific to LTI systems and defined by the eigenvalues' locations. It implies that the system's response neither grows unbounded nor decays, often resulting in sustained oscillations.


Lyapunov Stability: A more general concept applicable to all systems, ensuring that trajectories remain close to the equilibrium point if they start close enough.



\subsection{Stability of autonomous LTI}

\subsection*{\underline{Autonomous Systems}}

A system is considered \underline{\textbf{autonomous}} if its evolution depends only on time.\\

\underline{Example}

\[
\dot{x} = Ax
\]

\subsubsection{Diagonal matrices}

Let's introduce a trick for autonomous Linear Time-Invariant (LTI) systems.

First of all, recall the properties of a diagonal matrix and eigen-decomposition.
\begin{itemize}
    \item \textbf{Diagonal matrix}
        \[\dot{z} = Dz\]
        \[
        \begin{cases}
            \dot{x}_1 = d_1 x_1 \\
            \dots \\
            \dot{x}_n = d_n x_n
        \end{cases}
        \]
        The solution of each of the equations is: \(x_i = C_i e^{d_i t}\).
        So, the system is asymptotically stable when for all \(i\), \(d_i < 0\). 
        The system is stable when \(d_i \leq 0\).
    \item \textbf{Eigen-decomposition}\\
        We can represent the matrix as \(A = VDV^{-1}\), where \(D\) is a diagonal matrix.
\end{itemize}

Given an autonomous Linear Time-Invariant (LTI) system, let's switch to the system with a diagonal matrix:

\[
\dot{x} = Ax
\]

\[
\dot{x} = VDV^{-1}x
\]

\[
V^{-1}\dot{x} = V^{-1}VDV^{-1}x = DV^{-1}x
\]

Change of variables: \(z = V^{-1}x\), \(\dot{z} = Dz\), 

\[
V^{-1}\dot{x} = Dz
\]

The system is asymptotically stable when for all the elements of \(D\) are \( < 0\). 
The system is stable when the elements of \(D\) are \(\leq 0\).  

\subsubsection{Upper triangular matrices}

\begin{tcolorbox}[colback=white]
    Eigenvalues of upper triangular matrices are the diagonal elements.
\end{tcolorbox}
    

\subsubsection{General case}

Consider the LTI:
\[\dot x = Ax\]

\begin{tcolorbox}[colback=green!10,colframe=green!50!black]
    The system is called \textbf{stable} iff real parts of eigenvalues of A are non-positive.
\end{tcolorbox}

\begin{tcolorbox}[colback=green!10,colframe=green!50!black]
    The system is called \textbf{asymptotically stable} iff real parts of eigenvalues of A are  strictly negative. 
\end{tcolorbox}