\section{Lyapunov theory}

Stability is not defined only for linear systems, but for nonlinear systems as well.

There exist a small neighbourhood. Every solution that starts from this neighbourhood will asymptotically approach the same solution. 

\textbf{Asymptotic stability criteria:}\\
Autonomous dynamic system $\dot x = f(x)$ is asymptotically stable if there exists a scalar function $V = V(x) > 0$. 

% Autonomous - only \frac{\partial epen}{\partial s}on its state, does not depend on any external input. 

Note that computing $\dot V(x)$ comes along with computing $\dot x$. 
We can think of $V(x)$ as energy (pseudo-energy).

Asymptotic stability means that the system converges, and marginal stability means that the system does not diverge. 


\begin{tcolorbox}[colback=blue!10,colframe=blue!50!black,title=\textbf{Asymptotic stability}]
The system is called asymptotically stable if there exists a function $V > 0$ such that $\dot{V} < 0$, except for $V = 0, \dot{V} = 0$. 
\end{tcolorbox}

\begin{tcolorbox}[colback=green!10,colframe=green!50!black,title=\textbf{Marginal stability}]
The system $\dot{x} = f(x)$ is stable in the sense of Lyapunov if there exists $V > 0$ such that $\dot{V} \leq 0$. 
\end{tcolorbox}

\begin{tcolorbox}[colback=red!10,colframe=red!50!black,title=\textbf{Lyapunov function}]
A function $V > 0$ in this case is called a Lyapunov function. 
\end{tcolorbox}



\textbf{Example 1}
Consider the following system:
\[\dot{x} = -x\]

Let's introduce a function $V = x^2 > 0$ for all $x \neq 0$. $\dot{V} = 2x \cdot \dot{x} = 2x (-x) = -2x^2 < 0$. 
$V$ satisfies the Lyapunov criteria, so the system is (asymptotically) stable. 

\textbf{Example 2}
Consider the equation of the pendulum:
\[\ddot{q} = -\dot{q} - \sin(q)\]

% Let's introduce a function $V = q^2$.

\subsection{LaSalle's Principle}
The system is called stable if 


LaSalle's principle allows us to prove asymptotic stability even for $\dot{V}(x) \leq 0$ only for the trivial solution.


\subsection{Linear case}
Lyapunov theory applies for both nonlinear and linear systems. 
For linear systems the following feature exist:

For a system $\dot x = Ax$, we can always choose a Lyapunov candidate function in the form:

\[V = x^TSx, \] where S is positive-definite
\[\dot V = \dot x^T S x + x^T S \dot x = (Ax)^T Sx + x^TSAx = x^TA^TSx + x^TSAx = x^T (A^TS+SA) x\]


