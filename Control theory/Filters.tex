\section{Filters}

In literature, the system is frequently called a Plant, which in our case is actually a robot. 


\begin{tcolorbox}[colback=green!10,colframe=green!50!black,title=\textbf{Static system}]
    Linear systems for which the relation between input and output is constant are called \textbf{static systems}.
\end{tcolorbox}


\underline{Example}

(Laplace domain) $Y(s) = 10X(s)$. 

Note that when we are talking about independency on time not the signal, but the relation has to be time-independent. 


\begin{tcolorbox}[colback=green!10,colframe=green!50!black,title=\textbf{Dynamic system}]
    Dynamic systemns are linear systems for whhich the relation between input and output depends on time. 
\end{tcolorbox}

\underline{Dynamic system} 
\underline{Example 1}
State-space plant:
\[
\begin{cases}
    \dot x = Ax + Bu \\
    y = Cx 
\end{cases}    
\]

Apparently, the output $y$ does not depend on the input $u$. 

\underline{Example 2}

Laplace plant:
$Y(S) = \frac{1}{s^2 + 2s + 7} X(s)$
Time does not appear in this equaiton, but if we rewrite it into ode form:
\[\ddot y + 2 \dot y + 7 y = x\]

\underline{Example 3}
ODE plant:
\[\ddot y + 5\dot y + y = u\]



For these systems, the output depends not only on the current (time-wise) value of input, but on the entire history of input values.


Static systems form algebraic linear equaitons. Dynamical systems create linear differential equaitons

Dynamic systems have \textbf{state}, which changes over the time. 

For the system we can create the controller and Luenbberger observer:

% D_k is a fit-through term

% \[\]

% \begin{itemize}
%     \item High-pass filters
%     \item Low-passs filters
% \end{itemize}
