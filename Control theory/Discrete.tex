\section{Discrete dynamics}

Discrete case for dynamics:
\[x_{i+1} = Ax_i +  Bu_i\]

Such an equation is easy to work with: no derivatives are used here, it is easy to simulate. 

For this system, we can propose the control: \[u_i = -Kx_i + u_i*\]
\subsection{Stability}

Let's consider the system:
\[x_{i+1} = A x_i\]

\[x_{i+1} = V^{-1}DV x_i\]
\[V x_{i+1} = V V^{-1}DV x_i\]

The change of variables:
$z_{i} = V x_{i}$, $z_{i+1} = V x_{i+1}$

\[z_{i+1} = DV z_i\]

% The system will converge to 0 iff the eigenvalues of D: $\lambda_j < 1$. 
Let's consider the following example:
\[
\begin{bmatrix}
\vec{x}_{1,i+1} \\
x_{2, i+1}
\end{bmatrix}
=
\begin{bmatrix}
\alpha & -\beta \\
\beta & \alpha
\end{bmatrix}
\begin{bmatrix}
x_{1, i} \\
x_{2, i}
\end{bmatrix}
\]

Let's find the norms of \(x_{i+1}\) and \(x_i\).

\[
\| \begin{bmatrix}
x_{1, i} \\
x_{2, i}
\end{bmatrix}
\|^2 = x_{1, i}^2 + x_{2, i}^2
\]

\[
\| \begin{bmatrix}
x_{1, i+1} \\
x_{2, i+1}
\end{bmatrix}
\|^2 = \left| \begin{bmatrix}
\alpha x_{1, i} - \beta x_{2, i} \\
\beta x_{1, i} + \alpha x_{2, i}
\end{bmatrix} \right|^2
= (\alpha x_{1, i} - \beta x_{2, i})^2 + (\beta x_{1, i} + \alpha x_{2, i})^2
= \beta^2 (x_{2,i}^2 + x_{1,i}^2) + \alpha^2 (x_{2,i}^2 + x_{1,i}^2)
= (\alpha^2 + \beta^2) (x_{1, i}^2 + x_{2, i}^2)
\]

For stability we introduce:

\textbf{Continuous case:} \underline{Hurwitz matrix}. \textbf{Discrete case:} \underline{Schur matrix}.

\begin{tcolorbox}[colback=green!10,colframe=green!50!black,title=\textbf{Dynamical Systems}]
    \begin{itemize}
        \item Discrete system \(x_{i+1} = Ax_{i}\) is \textbf{stable} if and only if the absolute values of \(A\)'s eigenvalues are less than or equal to 1: \(|\lambda_i(A)| \leq 1\).
        \item Discrete system \(x_{i+1} = Ax_{i}\) is \textbf{asymptotically stable} if and only if the absolute values of \(A\)'s eigenvalues are less than 1: \(|\lambda_i(A)| < 1\).
    \end{itemize}
\end{tcolorbox}

\subsection{Analytical solution of Continuous LTI}

\subsubsection*{\underline{Case: $\dot{x} = ax(t)$}}



Recall that we can describe exponent as a series:
\[e^{a} = 1 + a + \frac{a^2}{2} + \frac{a^3}{6} + \dots\]
\[e^{A} = I + A + \frac{AA}{2} + \frac{AAA}{6} + \dots\]

Suppose that the solution of the dynamical system is:
\[x(t) = e^{at} x(0)\]
\[x(t) = e^{At} x(0)\]
Let's rewrite the exponent:
\[x(t) = (I + At + \frac{AAt^2}{2} + \frac{AAAt^3}{6} + \dots) x(0)\]
\[\dot x(t) = (A + AAt + \frac{AAAt^2}{2} + \dots) x(0)\]
\[\dot x(t) = A(I + At + \frac{AAt^2}{2} + \dots) x(0)\]


\[\dot x(t) = A e^{At} x(0)\]
\[\dot x(t) = A x(t)\]


\subsubsection*{\underline{Case: $\dot{x} = ax(t) + bu(t)$}}


\[
\dot{x} = ax(t) + bu(t)
\]
\[
\dot{x} e^{-at} - ae^{-at} x(t) = b e^{-at} u(t)
\]
\[
\frac{d}{dt} (x e^{-at}) = b e^{-at} u(t)
\]
\[
\int_{0}^{t} \frac{d}{d\tau}(e^{-a(t-\tau)}x(\tau)) d\tau = \int_{0}^{t} \frac{d}{d\tau}(e^{-a(t-\tau)}b u(\tau)) d\tau
\]
\[e^{-at} x(t) - x(0) = \int_{0}^{t} e^{-a\tau} bu(\tau) d\tau\]
\[x(t) = e^{at} x(0) + e^{at} \int_{0}^{t} e^{-a\tau} bu(\tau) d\tau\]

\[x(t) = e^{at} x(0) + \int_{0}^{t} e^{a(t-\tau)} bu(\tau) d\tau\]

\begin{tcolorbox}[colback=white]
\[x(t) = e^{At} x(0) + \int_{0}^{t} e^{A(t-\tau)} Bu(\tau) d\tau\]
\end{tcolorbox}


\subsection{Discretization}
\[
\begin{cases}
    x_0 = x(0) \\
    x_1 = x(\Delta t) \\
    x_2 = x(2\Delta t) \\
    \dots \\
    x_n = x(n\Delta t)
\end{cases}
\]

\[\dot{x} = \frac{1}{\Delta t}(x_{i+1} - x_i)\]
\[Ax_i= \frac{1}{\Delta t}(x_{i+1} - x_i)\]
\[x_{i+1} = Ax_i \Delta t + x_i\]

\begin{tcolorbox}
\[x_{i+1} = (A\Delta t + I) x_i\]
\end{tcolorbox}

or

\begin{tcolorbox}
\[x_{i+1} = (I - A\Delta t)^{-1} x_i\]
\end{tcolorbox}

Thus, the discrete state matrix is: \(A_d = A\Delta t + I\)
The control matrix is: \(B\Delta t\).