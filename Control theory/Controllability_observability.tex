
\noindent\fbox{%
\begin{minipage}{\textwidth}
\begin{tcolorbox}[controllability,title=Controllability]
A system is controllable on \(t_0 \leq t \leq t_f\) if it is possible to find a control input \(u(t)\) that would drive the system to a desired state \(x(t_f)\) from any initial state \(x_0\).
\end{tcolorbox}

\begin{tcolorbox}[observability,title=Observability]
A system is observable on \(t_0 \leq t \leq t_f\) if it is possible to exactly estimate the state of the system \(x(t_f)\), given any initial estimation error.
\end{tcolorbox}

\begin{tcolorbox}[alternative,title=Observability (Alternative)]
A system is observable on \(t_0 \leq t \leq t_f\) if any initial state \(x_0\) is uniquely determined by the output \(y(t)\) on that interval.
\end{tcolorbox}
\end{minipage}%
}


Observability criterion



Consider discrete LTI.

And a Luenberger observer:

\[\]

Error dynamics for the observer. 


If the system is controllable, we would be able to find such controller to the system, whcih would make the system stable, which means that the error goes to zero.


\[O^T = [C^T (A^T)C^T]\]


Observability matrix

error dynamics need to be stabilized, which means thta it needs t be controllable.

Observability criterion.


Needs to be full column rank.



\section{PBH controllability criterion}