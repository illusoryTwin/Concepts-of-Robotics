% Tha main tasks of control theory  include control design, trajectory tracking, point-to-point control

% \subsection*{Control design}
% With the control we choose we can make the system stable. 

% \begin{tcolorbox}[colback=green!10,colframe=green!50!black,title=\textbf{Dynamical systems}]
%     The task of \textbf{stabilizing control} is defining the control law that makes a certain solution of 
%     a dynamical system stable.
% \end{tcolorbox}

% Along with linear control, this is also true for non-linear. 

% Consider LTI system. 
% \[\dot x = Ax + Bu\]
% and choose control as a linear function of input:
% \[u = - Kx, \] where K is a \textbf{control gain}. 

% \begin{center}
%     \textbf{\textit{Stability condition}}
% \end{center}


% Then, the dynamical system could be represented as follows:
% \begin{equation}
%     \dot x = Ax - BKx = (A-BK)x
% \end{equation}


% The equation above describes autonomous closed loop system. 

% \noindent\fbox{
% \textbf{Hurwitz matrix}:
% A square matrix M that has eigenvalues with strictly negative \\

% real parts is called \textbf{Hurwitz}
% }



% \begin{tcolorbox}[colback=yellow!10,colframe=yellow!50!black,title=\textbf{Stability condition}]
%     The system is \textbf{asymptotically stable} if the eigenvalues' real parts of the matrix (A-BK) are strictly negative.
%     The system is \textbf{stable in the sense of Lyapunov} if the eigenvalues' real parts of the matrix (A-BK) are non-positive.
% \end{tcolorbox}

% Note that:
% \[\dot x = (a - bk) x\]

% The solution to this closed-loop system is:
% \[x(t) = x_0 e^{(a-bk)t}\]

% The system is converging to 0 for $a-bk<0$. 

% \subsection*{Trajectory tracking}
% The task is to stabilize the reference trajectory (system) around the points determined by  this equaiton

% The function $x* = x*(t)$ and control law $u* = u*(t)$ are a solution for the system $\dot x = Ax + Bu$. 
% This means that:
% \[\dot x* = Ax* + Bu*\]

% Let's propose a control law to stabilize the system. But first, let's define the distances between $\dot x*$ and $\dot x$

% \[\dot x* - \dot x = A(x* - x) + B(u* - u)\]

% Change of variables: $e=x* - x$, $u* - u$

% \[\dot e = Ae + Bv\]

% The equation above is called the \textbf{error dynamics}

% For making the error dynamics system converge to 0 (the motivation behind it is obvious), we need to propose 
% stabilizing control. 

% Suggest $v = -Ke$, then: 

% \[\dot e = Ae - BKe = (A-BK)e\]

% From these equations:
% \[\dot x* - \dot x = A(x* - x) + B(u* - u)\]
% \[\dot e = Ae - BKe\]

% notice that $-Ke = (u* - u)$, or 
% \[-K(x* - x) = (u* - u)\]

% Thus, 
% \[u = u* + K(x*-x)\]

% \subsection*{Point-to-Point Control}

% Point-to-point control is slightly different from Trajectory tracking  in terms of reference input - now it is $x* = const$ 
% and feed-forward control - now it is $u* = const$. 
% And thus, 
% \[Ax* + Bu* = 0\]
% We can find u* if it is not provided:
% \[u* = -B^{+}Ax*\]

% Since the error dynamics and the stabilizing control are the same as those in Trajectory planning, 
% we can consider the following control law:
% \[u = K(x*(t) - x) + u*(t)\]

% \[\dot x = Ax + BK(x*(t) - x) + Bu*(t)\]
% \[\dot x = (A - BK)x + BK x*(t) + Bu*(t)\]



\section{Control}

The main tasks of control theory include \textbf{\underline{control design}}, \textbf{\underline{trajectory tracking}}, and \textbf{\underline{point-to-point control}}.


\subsection{Control Design}

\begin{tcolorbox}[colback=green!10,colframe=green!50!black,title=\textbf{Dynamical Systems}]
    The task of \textbf{stabilizing control} is defining the control law that makes a certain solution of 
    a dynamical system stable.
\end{tcolorbox}

This is true for both linear and nonlinear systems.

Consider a Linear Time-Invariant (LTI) system:
\[\dot{x} = Ax + Bu,\]
and choose the control as a linear function of the state:
\[u = -Kx,\] where \(K\) is the \textbf{control gain}.

\begin{center}
    \textbf{\textit{Stability Condition}}
\end{center}

Then, the closed-loop system can be represented as:
\[\dot{x} = (A - BK)x.\]

The system is asymptotically stable if the eigenvalues of the matrix \(A - BK\) have strictly negative real parts.

\subsection{Trajectory Tracking}

The task is to stabilize the system around a reference trajectory.

Let \(x^*(t)\) and \(u^*(t)\) be solutions for the system \(\dot{x} = Ax + Bu\). This means that:
\[\dot{x}^* = Ax^* + Bu^*.\]

Define the error as \(e = x^* - x\) and \(v = u^* - u\).

Then, the error dynamics become:
\[\dot{e} = Ae + Bv.\]

To stabilize the system, suggest \(v = -Ke\), then:
\[\dot{e} = (A - BK)e.\]

Therefore, the control law becomes:
\[u = u^* + K(x^* - x).\]

\subsection{Point-to-Point Control}

Point-to-point control differs from trajectory tracking in that the reference input is constant, \(x^* = \text{const}\), and feed-forward control is also constant, \(u^* = \text{const}\).

Since the error dynamics and the stabilizing control are the same as in trajectory tracking, the control law becomes:
\[u = K(x^* - x) + u^*.\]

The dynamics of the system become:
\[\dot{x} = (A - BK)x + BKx^* + Bu^*.\]