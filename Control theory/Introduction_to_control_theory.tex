\documentclass{article}

% Packages for formatting
\usepackage[utf8]{inputenc}
\usepackage[T1]{fontenc}
\usepackage{geometry}
\usepackage{titling}
\usepackage{enumitem}
\usepackage{amsmath}
\usepackage{amsfonts}
\usepackage{amssymb}
\usepackage{graphicx}
\usepackage{hyperref}
\usepackage{fancyhdr}
\usepackage{lipsum} % For generating dummy text, you can remove this in your final document
\usepackage{xcolor}
\usepackage{tcolorbox}

% Define page geometry
\geometry{a4paper, margin=1in}

% Define title
\title{Study Material Template}
\author{Your Name}
\date{\today}

% Define header and footer
\pagestyle{fancy}
\fancyhf{}
\rhead{\thepage}
\lhead{\theauthor}
\chead{\thetitle}
\renewcommand{\headrulewidth}{1pt}
\renewcommand{\footrulewidth}{0pt}

% Define colors
\definecolor{controllable}{RGB}{50, 132, 191}
\definecolor{observable}{RGB}{65, 168, 95}
\definecolor{observalternative}{RGB}{191, 97, 106}

% Define tcolorbox styles
\tcbset{
    commonstyle/.style={
        boxrule=0.5mm,
        colback=white,
        arc=2mm,
        fonttitle=\bfseries
    },
    controllability/.style={
        commonstyle,
        colframe=controllable
    },
    observability/.style={
        commonstyle,
        colframe=observable
    },
    alternative/.style={
        commonstyle,
        colframe=observalternative
    }
}

\begin{document}

% Title
\maketitle

% Table of contents
\tableofcontents
\newpage

% Start of content

\section{Maths}

\section{Introduction}
% Your introduction here

\section{Topic 1}
\subsection{Subtopic 1.1}
% Your content here

\subsection{Subtopic 1.2}
% Your content here

\section{LQR}

\subsection{Ituition behind poles}
In control theory, the system state-space equation

\[\dot{x} = Ax+Bu\]

\[y=Cx+Du\]

has the transfer function

\[G(s)=C(sI-A)^{-1}B+D.\]

Since \((sI-A)^{-1}=\text{adj}(sI-A)\det(sI-A)\), where \(\text{adj}(sI-A)\) is the adjugate of \(sI-A\), the poles of \(G(s)\) are the numbers that satisfy \(\det(sI-A)=0\). This is exactly the characteristic equation of matrix \(A\), whose solutions are the eigenvalues of \(A\).

\subsection{LQR}
In pole-placement method, we want to place the poles in the specific spots (or, we choose specific eigenvalues). But it is not intuitive where to place them, especially for complex systems, systems with numerous actuators.
So, the new method is proposed. The key concept of the method lies in optimization of choosing \(K\). 

In LQR we find an optimal \(K\) by choosing parameters that are important to us, specifically how well the system performs and how much effort it takes to reach this performance.

If \(Q>>R\), then we are turning the problem of 
Let \(J\) be an additive cost function:
\[J(x_0, p(x, t)) = \int_{0}^{\infty} g(x, u)\]
\(Q\) - how bad if \(x\) is not where it is supposed to be.
\(Q\) - nonnegative, positive semidefinite.

if the system is a positions, velocity, and \(Q = \begin{bmatrix} 1 & 0 & 0 & 0 \\ 0 & 1 & 0 & 0 \\ 0 & 0 & 10 & 0 \\ 0 & 0 & 0 & 100 \end{bmatrix}\) we penalize for 

Suppose there is the best control law:
\[u = -kx\]

that minimizes the quadratic cost function.

\[J = \int x^T Q x + u^T R u\]

\textbf{Hamiltonian-Jacobi-Bellman (HJB)}

\[\min_u [g(x, u) + \frac{dJ}{dx f(x, u)}] = 0\]

Cost on effectiveness and energy to reach this effectiveness.

\subsection{Subtopic 2.1}
% Your content here

\subsection{Subtopic 2.2}
% Your content here

% Add more sections and subsections as needed

% End of content

\section{Controllability. Observability}
\include{controllability_observability.tex}




\section{Kalman Filter}
CHECK! 

Without a Kalman filter, if your system is not observable, you won't be able to retrieve all the state variables xx from the output yy alone.
By using a Kalman filter, you can estimate all the state variables xx even if they are not all directly observable in yy. The filter uses the 
model of the system and the measurements to infer the values of the unobserved states.

% The way we pick our initial state estimate does not have a bias.

% Assume you could pick your initial state estimate $\hat x_0$ such that
% the initial state estimation error behaves as a random 
% variable sampled from a Gaussian distribution $x_0 ~ N(0, P_0)$

% Knowing mean 

% \[\]

% E[w_i] = 0

% All subsequent means will be $E[]$



% Let's compute the autocovariance $P_{i+1}$ knowing $P_i$ 

% \[P_{i+1} = E[] = E[(Ax)]\]


% We can assume thta the random process is uncorrelated with x, so 

% Since the two processes are uncorreleted then their covariance is 0.  

% \[P_{i+1} = E[A x_i x_i^T A^T + w_i w_i^T] \]

% Note that x_i^T A^T = P_i - a covariance of the previous step. 
% \[P_{i+1} = A P_i A^T + Q\]


% $v_i$ - the random noise sampled from thhe Gaussian distribution, which represents the sensor error. 
% \[v_i ~ N(0, R)\]


% $\hat x_{i+1}^-$ is the estimation before measurements - apriori estimate.

% \[x_i+1 \]

% $L_i$ is a control gain. 



% Apriori error \[x_{i+1}^- = x_{i+1} - \\hat x_{i+1}^-\]


% We claculate aposteriori covariance knowing our apriori covariance.


% The question is how to minimize those covariances.





% f we have all of the measurements
% up to and including time k available for use in our estimate of X k , then we can form
% an a posteriori estimate, which we denote as 2:. The "+" superscript denotes that
% the estimate is a posteriori. One way to form the a posteriori state estimate is to
% compute the expected value of x k conditioned on all of the measurements up to
% and including time k:
% 2; = E [ X k / y l ,y 2 , . . ., Y k ] = a posteriori estimate




% DERIVATIONOF THE DISCRETE-TIME KALMAN FILTER 125
% the estimate is a posteriori. One way to form the a posteriori state estimate is to
% compute the expected value of x k conditioned on all of the measurements up to
% and including time k:
% 2; = E [ X k / y l ,y 2 , . . ., Y k ] = a posteriori estimate (5.3)
% If we have all of the measurements before (but not including) time k available for
% use in our estimate of X k , then we can form an a praori estimate, which we denote
% as 2 ; . The "-"superscript denotes that the estimate is a priori. One way to form
% the a priori state estimate is to compute the expected value of 51, conditioned on
% all of the measurements before (but not including) time k:
% 2; = E [ X k l y l , y 2 , . ., Y k - l ] = a priori estimate (5.4)
% It is important to note that 2; and 2; are both estimates of the same quantity; they
% are both estimates of X k . However, 2 i is our estimate of Xk before the measurement
% Y k is taken into account, and 2: is our estimate of 21, after the measurement y k
% is taken into account. We naturally expect 2; to be a better estimate than 2 i ,
% because we use more information to compute 2;:
% 2 i =
% 2' k = estimate of Xk after we process the measurement at time k (5.5)
% If we have measurements after time k available for use in our estimate of X k , then
% we can form a smoothed estimate. One way to form the smoothed state estimate is
% to compute the expected value of x k conditioned on all of the measurements that
% are available:
% estimate of Xk before we process the measurement at time k
% ? k l k + N = E [ x k l Y l i Y 2 , ' . . , Y k , " ' , Y k + N ] = smoothed estimate (5.6)
% where N is some positive integer whose value depends on the specific problem that
% is being solved. If we want to find the best prediction of x k more than one time
% step ahead of the available measurements, then we can form a predicted estimate.
% One way to form the predicted state estimate is to compute the expected value of
% Xk conditioned on all of the measurements that are available:
% 2 k l k - M = E [ x k j y 1 , y 2 , - . .,y k - ~ ]= predicted estimate 


\section{Reference materials}

% https://isharifi.ir/teaching/2019/IoT/[Dan_Simon]_Optimal_State_Estimation_Kalman,_H_In(BookFi).pdf
% % https://www.researchgate.net/profile/Hamid-Nasir-2/publication/297838887_Stocking_and_stand_structure_of_Sungai_Kerang_mangroves_in_matang_peninsular_Malaysia/links/5c05e070299bf169ae3049c0/Stocking-and-stand-structure-of-Sungai-Kerang-mangroves-in-matang-peninsular-Malaysia.pdf#[0,{%22name%22:%22XYZ%22},0,601,0]

(Controllability. Observability)
% https://www.ece.rutgers.edu/~gajic/psfiles/chap5.pdf

\section{Linearization}
% \documentclass{article}

% % Packages for formatting
% \usepackage[utf8]{inputenc}
% \usepackage[T1]{fontenc}
% \usepackage{geometry}
% \usepackage{titling}
% \usepackage{enumitem}
% \usepackage{amsmath}
% \usepackage{amsfonts}
% \usepackage{amssymb}
% \usepackage{graphicx}
% \usepackage{hyperref}
% \usepackage{fancyhdr}
% \usepackage{lipsum} % For generating dummy text, you can remove this in your final document
% \usepackage{amsmath}
% \usepackage{xcolor}
% \usepackage{tcolorbox}
% % \documentclass{article}

% % Define page geometry
% \geometry{a4paper, margin=1in}

% % Define title
% \title{Study Material Template}
% \author{Your Name}
% \date{\today}

% % Define header and footer
% \pagestyle{fancy}
% \fancyhf{}
% \rhead{\thepage}
% \lhead{\theauthor}
% \chead{\thetitle}
% \renewcommand{\headrulewidth}{1pt}
% \renewcommand{\footrulewidth}{0pt}

% \begin{document}

% % Title
% \maketitle

% % Table of contents
% \tableofcontents
% \newpage

% % Start of content

% \section{Introduction}
% % Your introduction here

% \section{Topic 1}
% \subsection{Subtopic 1.1}
% % Your content here

% \subsection{Subtopic 1.2}
% % Your content here

% \section{LQR}

% \subsection{INtuition behind poles}
% In control theory, the system state-space equation

% ˙x=Ax+Bu

% y=Cx+Du

% has the transfer function


% % % Define tcolorbox styles
% % \tcbset{
% %     commonstyle/.style={
% %         boxrule=0.5mm,
% %         colback=white,
% %         arc=2mm,
% %         fonttitle=\bfseries
% %     },
% %     theoremstyle/.style={
% %         commonstyle,
% %         colframe=theoremcolor
% %     },
% %     definitionstyle/.style={
% %         commonstyle,
% %         colframe=definitioncolor
% %     }
% % }


% % \begin{tcolorbox}[theoremstyle,title=Theorem]
% % The transfer function of a system is given by
% % \[ G(s) = C(sI-A)^{-1}B + D \]
% % where \( (sI-A)^{-1} = \text{adj}(sI-A) \det(sI-A) \), and the poles of \( G(s) \) are the values of \( s \) that satisfy \( \det(sI-A) = 0 \). This is exactly the characteristic equation of matrix \( A \), whose solutions are the eigenvalues of \( A \).
% % \end{tcolorbox}


% % G(s)=C(sI−A)−1B+D.

% % Since (sI−A)−1=adj(sI−A)det(sI−A), where adj(sI−A) is the adjugate of sI−A, the poles of G(s) are the numbers that satisfy det(sI−A)=0. This is exactly the characteristic equation of matrix A, whose solutions are the eigenvalues of A

% % .

% \subsection{LQR}
% In pole-placement method, we want to place the poles in the specific spots (or, we choose specific eigenvalues). But it is not intuitive where to place them, especially for complex systems, systems with numerous actuators.
% So, the new method is proposed. The key concept of the method lies in optimization of choosing K. 

% In LQR we find an optimal $K$ by choosing parameters that are important to us, specifically how well the system performs and how much effort it takes to reach this performance.

% If $Q>>R$, then we are turning the problem of 
% Let J be an additive cost function:
% \[J(x_0, p(x, t)) = \int_{0}^{\inf} g(x, u)\]
% Q - how bad if x is not where it is supposed to be.
% Q- nonnegative, poisitive semidefinite.

% if the system is a postions, velocity, and Q = [1, 0, 0, 0], [0, 1, 0, 0], [0, 0, 10, 0], [0, 0, 0, 100] we penalizez for 

% Suppose there is the best control law:
% \[u = -kx\]

% that miimizes the quadratic cost function.

% \[J = int x^T Q x + u^T R u\]
% I\textbf{Hamiltonian-Jacobi-Bellman (HJB)}

% \[min u [g(x, u) + \frac{dJ}{dx f(x, u)}] = 0\]

% Cost on effectiveness and energy to reach this effectiveness.
% \subsection{Subtopic 2.1}
% % Your content here

% \subsection{Subtopic 2.2}
% % Your content here

% % Add more sections and subsections as needed

% % End of content

% \section{Controllability. Observability}

% % \textbf{Controllability} 
% % A system is controllable on $t_0\leqt t\leqt_f$  if it possible to find control input  $u(t)$ that would drive the system to a desired state $x(t_f)$ from nay initial state $x_0$

% % \textbf{Observability}
% % A system is observable on $t_0\leqt t\leqt_f$ if it is possible to estimate exactly the state of the system $x(t_f)$, given any intial estimation error.

% % \textbf{Observability, alternative}
% % A system is observable on $t_0\leqt t\leqt_f$  if any initial state $x_0$ is uniquely determined by output $y(t)$ on that interval.

% \noindent\fbox{%
% \begin{minipage}{\textwidth}
% \begin{tcolorbox}[controllability,title=Controllability]
% A system is controllable on \(t_0 \leq t \leq t_f\) if it is possible to find a control input \(u(t)\) that would drive the system to a desired state \(x(t_f)\) from any initial state \(x_0\).
% \end{tcolorbox}

% \begin{tcolorbox}[observability,title=Observability]
% A system is observable on \(t_0 \leq t \leq t_f\) if it is possible to exactly estimate the state of the system \(x(t_f)\), given any initial estimation error.
% \end{tcolorbox}

% \begin{tcolorbox}[alternative,title=Observability (Alternative)]
% A system is observable on \(t_0 \leq t \leq t_f\) if any initial state \(x_0\) is uniquely determined by the output \(y(t)\) on that interval.
% \end{tcolorbox}
% \end{minipage}%
% }



% \end{document}


\documentclass{article}

% Packages for formatting
\usepackage[utf8]{inputenc}
\usepackage[T1]{fontenc}
\usepackage{geometry}
\usepackage{titling}
\usepackage{enumitem}
\usepackage{amsmath}
\usepackage{amsfonts}
\usepackage{amssymb}
\usepackage{graphicx}
\usepackage{hyperref}
\usepackage{fancyhdr}
\usepackage{lipsum} % For generating dummy text, you can remove this in your final document
\usepackage{xcolor}
\usepackage{tcolorbox}

% Define page geometry
\geometry{a4paper, margin=1in}

% Define title
\title{Study Material Template}
\author{Your Name}
\date{\today}

% Define header and footer
\pagestyle{fancy}
\fancyhf{}
\rhead{\thepage}
\lhead{\theauthor}
\chead{\thetitle}
\renewcommand{\headrulewidth}{1pt}
\renewcommand{\footrulewidth}{0pt}

% Define colors
\definecolor{controllable}{RGB}{50, 132, 191}
\definecolor{observable}{RGB}{65, 168, 95}
\definecolor{observalternative}{RGB}{191, 97, 106}

% Define tcolorbox styles
\tcbset{
    commonstyle/.style={
        boxrule=0.5mm,
        colback=white,
        arc=2mm,
        fonttitle=\bfseries
    },
    controllability/.style={
        commonstyle,
        colframe=controllable
    },
    observability/.style={
        commonstyle,
        colframe=observable
    },
    alternative/.style={
        commonstyle,
        colframe=observalternative
    }
}

\begin{document}

% Title
\maketitle

% Table of contents
\tableofcontents
\newpage

% Start of content

\section{Introduction}
% Your introduction here

\section{Topic 1}
\subsection{Subtopic 1.1}
% Your content here

\subsection{Subtopic 1.2}
% Your content here

\section{LQR}

\subsection{INtuition behind poles}
In control theory, the system state-space equation

\[\dot{x} = Ax+Bu\]

\[y=Cx+Du\]

has the transfer function

\[G(s)=C(sI-A)^{-1}B+D.\]

Since \((sI-A)^{-1}=\text{adj}(sI-A)\det(sI-A)\), where \(\text{adj}(sI-A)\) is the adjugate of \(sI-A\), the poles of \(G(s)\) are the numbers that satisfy \(\det(sI-A)=0\). This is exactly the characteristic equation of matrix \(A\), whose solutions are the eigenvalues of \(A\).

\subsection{LQR}
In pole-placement method, we want to place the poles in the specific spots (or, we choose specific eigenvalues). But it is not intuitive where to place them, especially for complex systems, systems with numerous actuators.
So, the new method is proposed. The key concept of the method lies in optimization of choosing \(K\). 

In LQR we find an optimal \(K\) by choosing parameters that are important to us, specifically how well the system performs and how much effort it takes to reach this performance.

If \(Q>>R\), then we are turning the problem of 
Let \(J\) be an additive cost function:
\[J(x_0, p(x, t)) = \int_{0}^{\infty} g(x, u)\]
\(Q\) - how bad if \(x\) is not where it is supposed to be.
\(Q\) - nonnegative, positive semidefinite.

if the system is a positions, velocity, and \(Q = \begin{bmatrix} 1 & 0 & 0 & 0 \\ 0 & 1 & 0 & 0 \\ 0 & 0 & 10 & 0 \\ 0 & 0 & 0 & 100 \end{bmatrix}\) we penalize for 

Suppose there is the best control law:
\[u = -kx\]

that minimizes the quadratic cost function.

\[J = \int x^T Q x + u^T R u\]

\textbf{Hamiltonian-Jacobi-Bellman (HJB)}

\[\min_u [g(x, u) + \frac{dJ}{dx f(x, u)}] = 0\]

Cost on effectiveness and energy to reach this effectiveness.

\subsection{Subtopic 2.1}
% Your content here

\subsection{Subtopic 2.2}
% Your content here

% Add more sections and subsections as needed

% End of content

\section{Taylor series approximation}


\end{document}


\end{document}
