\documentclass{article}

% Packages for formatting
\usepackage[utf8]{inputenc}
\usepackage[T1]{fontenc}
\usepackage{geometry}
\usepackage{titling}
\usepackage{enumitem}
\usepackage{amsmath}
\usepackage{amsfonts}
\usepackage{amssymb}
\usepackage{graphicx}
\usepackage{hyperref}
\usepackage{fancyhdr}
\usepackage{lipsum} % For generating dummy text, you can remove this in your final document
\usepackage{xcolor}
\usepackage{tcolorbox}

% Define page geometry
\geometry{a4paper, margin=1in}

% Define title
\title{Study Material Template}
\author{Your Name}
\date{\today}

% Define header and footer
\pagestyle{fancy}
\fancyhf{}
\rhead{\thepage}
\lhead{\theauthor}
\chead{\thetitle}
\renewcommand{\headrulewidth}{1pt}
\renewcommand{\footrulewidth}{0pt}

% Define colors
\definecolor{controllable}{RGB}{50, 132, 191}
\definecolor{observable}{RGB}{65, 168, 95}
\definecolor{observalternative}{RGB}{191, 97, 106}

% Define tcolorbox styles
\tcbset{
    commonstyle/.style={
        boxrule=0.5mm,
        colback=white,
        arc=2mm,
        fonttitle=\bfseries
    },
    controllability/.style={
        commonstyle,
        colframe=controllable
    },
    observability/.style={
        commonstyle,
        colframe=observable
    },
    alternative/.style={
        commonstyle,
        colframe=observalternative
    }
}

\begin{document}

% Title
\maketitle

% Table of contents
\tableofcontents
\newpage

% Start of content

The main tasks of Control Theory are 

\begin{enumerate}
    \item Stabilization
    \item Tracking 
\end{enumerate}
\section{Maths}

\section{Introduction}

\section{Introduction}
\begin{tcolorbox}[colback=green!10,colframe=green!50!black,title=\textbf{Dynamical systems}]
    Let's consider the \(n\)-th order ordinary differential equation (ODE):
    \[
    \mathbf{x}^{(n)} = \mathbf{f}(\mathbf{x}^{(n-1)}, \mathbf{x}^{(n-2)}, ..., \ddot{\mathbf{x}}, \dot{\mathbf{x}}, \mathbf{x}, t),
    \]
    where \(x(t)\) is a solution for the system, and \(t\) is an independent variable (usually time).
    This equation represents the dynamics of the system and it is called a \textbf{dynamical system}.\\
    x is called the \textbf{state} of the dynamical system.
\end{tcolorbox}


In canonical form, linear ODE is represented in the following way:
\[a_{n}z^{(n)} +a_{n-1}z^{(n-1)}+...+a_{2}\ddot z+a_{1}\dot z + a_0 z= b_0\]

The set $\{ \mathbf{x}, \ \dot{\mathbf{x}} \ ..., \ \mathbf{x}^{(n-1)} \}$ is called the \textbf{state  of the system}.

\textbf{State} of the system is a minimal set of variables that describe the system. Based on the current state and future inputs,
we can predict the behaviour of the system. \\

\subsection{Introducing input}
General form of an n-th order linear ODE with an input can be presented as follows:
%
\begin{equation}
    a_n y^{(n)} + 
    ... +
    a_2 \ddot{y} + a_1 \dot{y} + 
    a_0 y = u(t)
\end{equation}

\bigskip

State-space representation of a linear system with an input is:
%
\begin{equation}
    \dot x = Ax + Bu
\end{equation}

$A$ is called a \textbf{state matrix} and x is a \textbf{state vector}, 

$B$ is called a \textbf{control matrix} and $u$ is a \textbf{control vector}.

$u$ might be either a scalar or a vector.

\subsection{Introducing output}
Equations might also have an output, which can have plenty of physical meanings and interpretations. Let's list some of them:
what we measure (position and orientation of a motor), what we want to control (the height of the quadrotor).

Output is usually defined as $y$. 

Example of system wth input and output:

\begin{equation}
    \begin{cases}
    \mathbf{\dot{x}}=\mathbf{A}\mathbf{x} + \mathbf{B}\mathbf{u} \\
    \mathbf{y}=\mathbf{C}\mathbf{x}
    \end{cases}
\end{equation}

If $u$ and $y$ are scalars, the system is called \emph{single-input single-output (SISO)}, if they are vectors - \emph{multi-input multi-output (MIMO)}.

\subsection*{Linear systems}
In case if relationships between state, output and control are linear, we can model system in following form:
\begin{equation}
\begin{cases}
\mathbf{\dot{x}} =\mathbf{A}\mathbf{x} + \mathbf{B}\mathbf{u} \\
\mathbf{y}=\mathbf{C}\mathbf{x} + \mathbf{D}\mathbf{u}
\end{cases}
\end{equation}


Where
\begin{itemize}
    \item $\mathbf{x} \in \mathbb{R}^n$: states of the system
    \item $\mathbf{y} \in \mathbb{R}^l$: output vector
    \item $\mathbf{u} \in \mathbb{R}^m$: control inputs
    \item $\mathbf{A} \in \mathbb{R}^{n \times n}$: state matrix
    \item $\mathbf{B} \in \mathbb{R}^{n \times m}$: input matrix
    \item $\mathbf{C} \in \mathbb{R}^{l \times n}$: output matrix
    \item $\mathbf{D} \in \mathbb{R}^{l \times m}$: feedforward matrix
\end{itemize}


If matrices A, B, C, D are time-independent, then we call such systems \textbf{time-invariant.} 

More frequently we work with systems when the output does not depend on control. 

\begin{equation}
    \begin{cases}
    \mathbf{\dot{x}}=\mathbf{A}\mathbf{x} + \mathbf{B}\mathbf{u} \\
    \mathbf{y}=\mathbf{C}\mathbf{x}
    \end{cases}
    \end{equation}


    
\subsection{ODE to State-Space conversion}
\[
\dddot{y} + a_2 \ddot{y} + a_0 y = u
\]
\[
\begin{bmatrix}
\dot{y} \\
\ddot{y} \\
\dddot{y}
\end{bmatrix} = 
\begin{bmatrix}
0 & 1 & 0 \\
0 & 0 & 1 \\
-a_0 & -a_1 & -a_2
\end{bmatrix}
\begin{bmatrix}
y \\
\dot{y} \\
\ddot{y}
\end{bmatrix} + 
\begin{bmatrix}
0 \\
0 \\
u
\end{bmatrix}
\]

\subsection{State-Space to ODE conversion}

\[
\begin{cases}
\dot{x} = Ax \\
y = Cx
\end{cases}
\]

We need to represent this system as an ODE in the form:
\[y^{(n)} = d_{n-1}y^{(n-1)} + d_{n-2}y^{(n-2)} + \dots + d_1 \dot{y} + d_0 \]

Let's take the derivative of \(y\):
\[\dot{y} = C\dot{x} = CAx\]
\dots
\[y^{(n)} = CA^{(n)} x\]

\[y = \begin{bmatrix} C \\ CA \\ \dots \\ CA^{(n-1)} \end{bmatrix} x = Ox, \]

\(O\) is called the \textbf{observability matrix}.

\[x  = O^{-1} y \]
Then, 
\[y^{(n)} = CA^{(n)} x = CA^{(n)}O^{-1} y = CA^{(n)}O^{-1}  \begin{bmatrix} y \\ \dot{y} \\ \vdots \\ y^{(n-1)}\end{bmatrix}\]

% Your introduction here


\subsection{Critical Point (Node)}

\begin{tcolorbox}[colback=blue!10,colframe=blue!50!blue,title=\textbf{Critical Point (Node)}]
Consider the following LTI:
\[
\dot{x} = f(\mathbf{x}, t)
\]
\(x_0\) is called a \textbf{Node}, or \textbf{Critical Point}, if \(f(x_0) = 0\).
\end{tcolorbox}

\subsection{Stability}
A system is stable if:
\[
\|x(0) - x_0 \| < \delta \quad \| x(t) -x_0\| < \epsilon
\]

% We can think of it as: if the starting point is in the \(\delta\)-neighborhood of the node \(x_0\), the rest of the trajectory \(x(t)\) is in the \(\epsilon\)-neighborhood of the node.

% Or, the solutions starting from the \delta - ball do not diverge. 

% \subsection{Asymptotic Stability}

% A system is asymptotically stable if:
% \[
% \|x(0) - x_0 \| < \delta \quad \| x(t) -x_0\| < \epsilon
% \]


% Tha main tasks of control theory  include control design, trajectory tracking, point-to-point control

% \subsection*{Control design}
% With the control we choose we can make the system stable. 

% \begin{tcolorbox}[colback=green!10,colframe=green!50!black,title=\textbf{Dynamical systems}]
%     The task of \textbf{stabilizing control} is defining the control law that makes a certain solution of 
%     a dynamical system stable.
% \end{tcolorbox}

% Along with linear control, this is also true for non-linear. 

% Consider LTI system. 
% \[\dot x = Ax + Bu\]
% and choose control as a linear function of input:
% \[u = - Kx, \] where K is a \textbf{control gain}. 

% \begin{center}
%     \textbf{\textit{Stability condition}}
% \end{center}


% Then, the dynamical system could be represented as follows:
% \begin{equation}
%     \dot x = Ax - BKx = (A-BK)x
% \end{equation}


% The equation above describes autonomous closed loop system. 

% \noindent\fbox{
% \textbf{Hurwitz matrix}:
% A square matrix M that has eigenvalues with strictly negative \\

% real parts is called \textbf{Hurwitz}
% }



% \begin{tcolorbox}[colback=yellow!10,colframe=yellow!50!black,title=\textbf{Stability condition}]
%     The system is \textbf{asymptotically stable} if the eigenvalues' real parts of the matrix (A-BK) are strictly negative.
%     The system is \textbf{stable in the sense of Lyapunov} if the eigenvalues' real parts of the matrix (A-BK) are non-positive.
% \end{tcolorbox}

% Note that:
% \[\dot x = (a - bk) x\]

% The solution to this closed-loop system is:
% \[x(t) = x_0 e^{(a-bk)t}\]

% The system is converging to 0 for $a-bk<0$. 

% \subsection*{Trajectory tracking}
% The task is to stabilize the reference trajectory (system) around the points determined by  this equaiton

% The function $x* = x*(t)$ and control law $u* = u*(t)$ are a solution for the system $\dot x = Ax + Bu$. 
% This means that:
% \[\dot x* = Ax* + Bu*\]

% Let's propose a control law to stabilize the system. But first, let's define the distances between $\dot x*$ and $\dot x$

% \[\dot x* - \dot x = A(x* - x) + B(u* - u)\]

% Change of variables: $e=x* - x$, $u* - u$

% \[\dot e = Ae + Bv\]

% The equation above is called the \textbf{error dynamics}

% For making the error dynamics system converge to 0 (the motivation behind it is obvious), we need to propose 
% stabilizing control. 

% Suggest $v = -Ke$, then: 

% \[\dot e = Ae - BKe = (A-BK)e\]

% From these equations:
% \[\dot x* - \dot x = A(x* - x) + B(u* - u)\]
% \[\dot e = Ae - BKe\]

% notice that $-Ke = (u* - u)$, or 
% \[-K(x* - x) = (u* - u)\]

% Thus, 
% \[u = u* + K(x*-x)\]

% \subsection*{Point-to-Point Control}

% Point-to-point control is slightly different from Trajectory tracking  in terms of reference input - now it is $x* = const$ 
% and feed-forward control - now it is $u* = const$. 
% And thus, 
% \[Ax* + Bu* = 0\]
% We can find u* if it is not provided:
% \[u* = -B^{+}Ax*\]

% Since the error dynamics and the stabilizing control are the same as those in Trajectory planning, 
% we can consider the following control law:
% \[u = K(x*(t) - x) + u*(t)\]

% \[\dot x = Ax + BK(x*(t) - x) + Bu*(t)\]
% \[\dot x = (A - BK)x + BK x*(t) + Bu*(t)\]



\section{Control}

The main tasks of control theory include \textbf{\underline{control design}}, \textbf{\underline{trajectory tracking}}, and \textbf{\underline{point-to-point control}}.


\subsection{Control Design}

\begin{tcolorbox}[colback=green!10,colframe=green!50!black,title=\textbf{Stabilizing control}]
    The task of \textbf{stabilizing control} is defining the control law that makes a certain solution of 
    a dynamical system stable.
\end{tcolorbox}

This is true for both linear and nonlinear systems.

Consider a Linear Time-Invariant (LTI) system:
\[\dot{x} = Ax + Bu,\]
and choose the control as a linear function of the state:
\[u = -Kx,\] where \(K\) is the \textbf{control gain}.

\begin{center}
    \textbf{\textit{Stability Condition}}
\end{center}

Then, the closed-loop system can be represented as:
\[\dot{x} = (A - BK)x.\]

The system is asymptotically stable if the eigenvalues of the matrix \(A - BK\) have strictly negative real parts.

Or, matrix $(A-BK) \in \mathcal{H}$ should be Hurwitz. 

\subsection{Trajectory Tracking}

The task is to stabilize the system around a reference trajectory.

Let \(x^*(t)\) and \(u^*(t)\) be solutions for the system \(\dot{x} = Ax + Bu\). This means that:
\[\dot{x}^* = Ax^* + Bu^*.\]

Define the error as \(e = x^* - x\) and \(v = u^* - u\).

Then, the error dynamics become:
\[\dot{e} = Ae + Bv.\]

To stabilize the system, suggest \(v = -Ke\), then:
\[\dot{e} = (A - BK)e.\]

Therefore, the control law becomes:
\[u = u^* + K(x^* - x).\]

\subsection{Point-to-Point Control}

Point-to-point control differs from trajectory tracking in that the reference input is constant, \(x^* = \text{const}\), and feed-forward control is also constant, \(u^* = \text{const}\).

Since the error dynamics and the stabilizing control are the same as in trajectory tracking, the control law becomes:
\[u = K(x^* - x) + u^*.\]

The dynamics of the system become:
\[\dot{x} = (A - BK)x + BKx^* + Bu^*.\]

\section{Topic 1}
\subsection{Subtopic 1.1}
% Your content here

\subsection{Subtopic 1.2}
% Your content here

\section{LQR}

\subsection{Intuition behind poles}
In control theory, the system state-space equation

\[\dot{x} = Ax+Bu\]

\[y=Cx+Du\]

has the transfer function

\[G(s)=C(sI-A)^{-1}B+D.\]

Since \((sI-A)^{-1}=\text{adj}(sI-A)\det(sI-A)\), where \(\text{adj}(sI-A)\) is the adjugate of \(sI-A\), the poles of \(G(s)\) are the numbers that satisfy \(\det(sI-A)=0\). This is exactly the characteristic equation of matrix \(A\), whose solutions are the eigenvalues of \(A\).

\subsection{LQR}
In pole-placement method, we want to place the poles in the specific spots (or, we choose specific eigenvalues). But it is not intuitive where to place them, especially for complex systems, systems with numerous actuators.
So, the new method is proposed. The key concept of the method lies in optimization of choosing \(K\). 

In LQR we find an optimal \(K\) by choosing parameters that are important to us, specifically how well the system performs and how much effort it takes to reach this performance.

If \(Q>>R\), then we are turning the problem of 
Let \(J\) be an additive cost function:
\[J(x_0, p(x, t)) = \int_{0}^{\infty} g(x, u)\]
\(Q\) - how bad if \(x\) is not where it is supposed to be.
\(Q\) - nonnegative, positive semidefinite.

if the system is a positions, velocity, and \(Q = \begin{bmatrix} 1 & 0 & 0 & 0 \\ 0 & 1 & 0 & 0 \\ 0 & 0 & 10 & 0 \\ 0 & 0 & 0 & 100 \end{bmatrix}\) we penalize for 

Suppose there is the best control law:
\[u = -kx\]

that minimizes the quadratic cost function.

\[J = \int x^T Q x + u^T R u\]

\textbf{Hamiltonian-Jacobi-Bellman (HJB)}

\[\min_u [g(x, u) + \frac{dJ}{dx f(x, u)}] = 0\]

Cost on effectiveness and energy to reach this effectiveness.

\subsection{Subtopic 2.1}
% Your content here

\subsection{Subtopic 2.2}
% Your content here

% Add more sections and subsections as needed

% End of content
\section{Observer}
% \section{Observer}


% When the full state feedback is unavailablle, we introducce an observer to \underline{estimate} the state $x$.

% % https://courses.engr.illinois.edu/ece486/fa2018/handbook/lec22.html



% We already know how to create a system with a controller, but how to check the current state of the system?

% \[
% \begin{cases}
%     \dot x = Ax + Bu\\
%     u = -Kx
% \end{cases}
% \]

% We can try to estimate it with the measurements, for example with sensors. But in real life 
% the task is not that trivial due to some problems:

% \begin{enumerate}
%     \item Lack of sensors. For a quadrotor we can not measure the height straight forwardly 
%     \item Measurements can be imprecise or biased
%     \item Measurements can be only made in discrete time
% \end{enumerate}

% The key problem arises when the output of the system $y$ is not the whole state $x$, but $y = Cx$, which
% means that we are able to get the state partially.

% These are just several problems that create the difficulties for us to measure the state of the system. \\

% \subsection{Measurement and estimation}

% \[
% \begin{cases}
%     \dot x = Ax + Bu \\
%     y = Cx \\
%     \hat x(t) = estimate  (y(t)) \\
%     u = -K \hat x
% \end{cases}
% \] 

% $x$ and $y$ are the state and output (actual or true).
% When we do not know the exact state x, we can only estimate it. We estimate x ($\hat x$) based on the history of y values. The control law is now governed by the estimated state $\hat x$. \\

% \begin{center}
%     \textit{\textbf{Estimation error}}
% \end{center}

% State estimation error is the following:
% \[\epsilon = \hat x - x\]

% \subsection{Dynamics estimation}

% We can always find $~ y = C \hat x - y = C \hat x - Cx = C \epsilon$. 

% Let's suggest that the dynamics should also hold for the observed state:
% \[\hat \dot x = A \hat x + Bu\]

% Let's introduce in our equation a linear correction law $-L ~y$. 
% Since $~y = C \hat x - y$, we get:

% \begin{equation}{1}
%     \hat \dot x = A \hat x + Bu + L(y - C \hat x) 
% \end{equation}

% This is called Luenberger observer.  

% But how to find suitable observer gain $L$?

% Let's subtract $\dot x = Ax + Bu$ from \equation{1}. The equation we got is the observer error dynamics:

% \begin{equation}
%     \hat \dot x - \dot x = A \hat x - Ax + L(y - C \hat x)
% \begin{equation}

% \begin{equation}
%     \hat \dot x - \dot x = A \hat x - Ax + L(Cx - C \hat x) = A \hat x - Ax + LC(x \hat x)
% \begin{equation}

% \begin{equation}
%     \dot \epsilon = A \epsilon - LC \epsilon = (A-LC) \epsilon
% \begin{equation}


\section{Observer}

When the full state feedback is unavailable, we introduce an observer to \underline{estimate} the state \(x\).

% https://courses.engr.illinois.edu/ece486/fa2018/handbook/lec22.html

We already know how to create a system with a controller, but how to check the current state of the system?

\[
\begin{cases}
    \dot x = Ax + Bu\\
    u = -Kx
\end{cases}
\]

We can try to estimate it with measurements, for example with sensors. But in real life, the task is not that trivial due to some problems:

\begin{enumerate}
    \item Lack of sensors. For a quadrotor, we cannot measure the height straightforwardly.
    \item Measurements can be imprecise or biased.
    \item Measurements can be only made in discrete time.
\end{enumerate}

The key problem arises when the output of the system \(y\) is not the whole state \(x\), but \(y = Cx\), which means that we are able to get the state partially.

These are just several problems that create difficulties for us to measure the state of the system.

\subsection{Measurement and estimation}

\[
\begin{cases}
    \dot x = Ax + Bu \\
    y = Cx \\
    \hat x(t) = \text{estimate}(y(t)) \\
    u = -K \hat x
\end{cases}
\]

\(x\) and \(y\) are the state and output (actual or true). When we do not know the exact state \(x\), we can only estimate it. We estimate \(x\) (\(\hat x\)) based on the history of \(y\) values. The control law is now governed by the estimated state \(\hat x\).

\begin{center}
    \textit{\textbf{Estimation error}}
\end{center}

State estimation error is the following:
\[
\epsilon = \hat x - x
\]

\subsection{Dynamics estimation}

We can always find \(y = C \hat x - y\).
\[ y = C \hat x - y = C \hat x - Cx = C \epsilon\]

Let's suggest that the dynamics should also hold for the observed state:
\[
\hat \dot x = A \hat x + Bu
\]

Let's introduce in our equation a linear correction law \( -L \, y\). Since \(y = C \hat x - y\), we get:

\begin{equation}\label{eq:luenberger_obs}
    \hat \dot x = A \hat x + Bu + L(y - C \hat x) 
\end{equation}


This is called the Luenberger observer.  

But how to find a suitable observer gain \(L\)?

Let's subtract \(\dot x = Ax + Bu\) from Equation  (\ref{eq:luenberger_obs}) . The equation we got is the observer error dynamics:

\begin{equation}
    \hat \dot x - \dot x = A \hat x - Ax + L(y - C \hat x)
\end{equation}

\begin{equation}
    \hat \dot x - \dot x = A \hat x - Ax + L(Cx - C \hat x) = A \hat x - Ax + LC(x - \hat x)
\end{equation}

\begin{equation}
    \dot \epsilon = A \epsilon - LC \epsilon = (A-LC) \epsilon
\end{equation}


With no knowledge of $x$ nad $\hat x$, we can define the stability of the system. 

What we want is the error converging to 0. To obtain this, the observer $\dot \epsilon = (A - LC) \epsilon$ needs to be stable. 
$A-LC \in \mathcal{H}$.

Recall:
\begin{itemize}
    \item Controller design: find such $K$ that $A-BK \in \mathcal{H}$.
    \item Observer design: find such $L$ that: $A-LC \in \mathcal{H}$
\end{itemize}

But now the gain $L$ is in the left side for the  observer (unlike $K$ for the controller), so we cannot use any stabilization methods
(LQR, pole placement) right away. 

We need to introduce the following change (or, we can solve the \textit{dual problem}):
find such L that:
\[A^T - C^TL^T \in \mathcal{H}\]

ANd  now, for this equation we can use LQR or poole-placement. 

% \dot x = Ax 

\subsection{Observer + Controller}

% \[
% \begin{cases}
%     \dot x = Ax + Bu \label{eq:state_eq} \\
%     y = Cx \label{eq:output_eq} \\
%     \hat x(t) = \text{estimate}(y(t)) \label{eq:estimate_eq} \\
%     u = -K \hat x \label{eq:control_eq}
% \end{cases}
% \]

% Let's substitute $u$ into (\ref{eq:state_eq}), $u$ and $y$ into (\ref{eq:state_eq})


\[
\begin{cases}
    \dot x = Ax + BK \hat x \\
    \hat \dot x = A \hat x - BK \hat x + LC(x - \hat x)
\end{cases}
\]


\begin{equation}
    \begin{cases}
        \dot x = Ax + BK \hat x \label{eq:state_eq1} \\
        \hat \dot x = A \hat x - BK \hat x - LC(\hat x - x) 
    \end{cases}
\end{equation}


% \[
% [\dot x \\ \hat \dot x] = [A & -BK \\ LC & A-BK-LC][x \\ \hat x]
% \]

Let's do a change of variables so that it would be easier to analyze the eigenvalues:
$e = x - \hat x$, $\dot e = \dot x - \hat \dot x$. 
Let's subtract (1) of  \ref{eq:state_eq1} from (2) of \ref{eq:state_eq1}:

\[\dot e = A \hat x - BK \hat x - LC(\hat x - x) - Ax + BK \hat x \]
\[\dot e = A e - LCe\]

\begin{equation}
    \dot e = (A-LC)e
\end{equation}

Now let's turn back to the equation 1 of the system and rewrite it. 
\[\dot  x = Ax - BK \hat x = Ax - BK(x-e)\]

\begin{equation}
    \dot  x = (A-BK)x + BKe
\end{equation}

\[
\begin{cases}
    \dot e = (A - LC) e \\
    \dot x = (A - BK)x + BKe
\end{cases}
\] 

\[
\begin{bmatrix}
    \dot x \\ 
    \dot e
\end{bmatrix} = 
\begin{bmatrix}
    A-BK & BK \\ 
    0 & A - LC
\end{bmatrix} 
\begin{bmatrix}
    x \\ 
    e
\end{bmatrix}
\]

The eigenvalues of the system are the union of eigenvalues of $(A-  BK)$ and $(A-LC)$. 


\section{Filters}
\section{Filters}

In literature, the system is frequently called a Plant, which in our case is actually a robot. 


\begin{tcolorbox}[colback=green!10,colframe=green!50!black,title=\textbf{Static system}]
    Linear systems for which the relation between input and output is constant are called \textbf{static systems}.
\end{tcolorbox}


\underline{Example}

(Laplace domain) $Y(s) = 10X(s)$. 

Note that when we are talking about independency on time not the signal, but the relation has to be time-independent. 


\begin{tcolorbox}[colback=green!10,colframe=green!50!black,title=\textbf{Dynamic system}]
    Dynamic systemns are linear systems for whhich the relation between input and output depends on time. 
\end{tcolorbox}

\underline{Dynamic system} 
\underline{Example 1}
State-space plant:
\[
\begin{cases}
    \dot x = Ax + Bu \\
    y = Cx 
\end{cases}    
\]

Apparently, the output $y$ does not depend on the input $u$. 

\underline{Example 2}

Laplace plant:
$Y(S) = \frac{1}{s^2 + 2s + 7} X(s)$
Time does not appear in this equaiton, but if we rewrite it into ode form:
\[\ddot y + 2 \dot y + 7 y = x\]

\underline{Example 3}
ODE plant:
\[\ddot y + 5\dot y + y = u\]



For these systems, the output depends not only on the current (time-wise) value of input, but on the entire history of input values.


Static systems form algebraic linear equaitons. Dynamical systems create linear differential equaitons

Dynamic systems have \textbf{state}, which changes over the time. 

For the system we can create the controller and Luenbberger observer:

% D_k is a fit-through term

% \[\]

% \begin{itemize}
%     \item High-pass filters
%     \item Low-passs filters
% \end{itemize}


\section{Controllability. Observability}
\include{controllability_observability.tex}




\section{Kalman Filter}
CHECK! 

Without a Kalman filter, if your system is not observable, you won't be able to retrieve all the state variables xx from the output yy alone.
By using a Kalman filter, you can estimate all the state variables xx even if they are not all directly observable in yy. The filter uses the 
model of the system and the measurements to infer the values of the unobserved states.

% The way we pick our initial state estimate does not have a bias.

% Assume you could pick your initial state estimate $\hat x_0$ such that
% the initial state estimation error behaves as a random 
% variable sampled from a Gaussian distribution $x_0 ~ N(0, P_0)$

% Knowing mean 

% \[\]

% E[w_i] = 0

% All subsequent means will be $E[]$



% Let's compute the autocovariance $P_{i+1}$ knowing $P_i$ 

% \[P_{i+1} = E[] = E[(Ax)]\]


% We can assume thta the random process is uncorrelated with x, so 

% Since the two processes are uncorreleted then their covariance is 0.  

% \[P_{i+1} = E[A x_i x_i^T A^T + w_i w_i^T] \]

% Note that x_i^T A^T = P_i - a covariance of the previous step. 
% \[P_{i+1} = A P_i A^T + Q\]


% $v_i$ - the random noise sampled from thhe Gaussian distribution, which represents the sensor error. 
% \[v_i ~ N(0, R)\]


% $\hat x_{i+1}^-$ is the estimation before measurements - apriori estimate.

% \[x_i+1 \]

% $L_i$ is a control gain. 



% Apriori error \[x_{i+1}^- = x_{i+1} - \\hat x_{i+1}^-\]


% We claculate aposteriori covariance knowing our apriori covariance.


% The question is how to minimize those covariances.





% f we have all of the measurements
% up to and including time k available for use in our estimate of X k , then we can form
% an a posteriori estimate, which we denote as 2:. The "+" superscript denotes that
% the estimate is a posteriori. One way to form the a posteriori state estimate is to
% compute the expected value of x k conditioned on all of the measurements up to
% and including time k:
% 2; = E [ X k / y l ,y 2 , . . ., Y k ] = a posteriori estimate




% DERIVATIONOF THE DISCRETE-TIME KALMAN FILTER 125
% the estimate is a posteriori. One way to form the a posteriori state estimate is to
% compute the expected value of x k conditioned on all of the measurements up to
% and including time k:
% 2; = E [ X k / y l ,y 2 , . . ., Y k ] = a posteriori estimate (5.3)
% If we have all of the measurements before (but not including) time k available for
% use in our estimate of X k , then we can form an a praori estimate, which we denote
% as 2 ; . The "-"superscript denotes that the estimate is a priori. One way to form
% the a priori state estimate is to compute the expected value of 51, conditioned on
% all of the measurements before (but not including) time k:
% 2; = E [ X k l y l , y 2 , . ., Y k - l ] = a priori estimate (5.4)
% It is important to note that 2; and 2; are both estimates of the same quantity; they
% are both estimates of X k . However, 2 i is our estimate of Xk before the measurement
% Y k is taken into account, and 2: is our estimate of 21, after the measurement y k
% is taken into account. We naturally expect 2; to be a better estimate than 2 i ,
% because we use more information to compute 2;:
% 2 i =
% 2' k = estimate of Xk after we process the measurement at time k (5.5)
% If we have measurements after time k available for use in our estimate of X k , then
% we can form a smoothed estimate. One way to form the smoothed state estimate is
% to compute the expected value of x k conditioned on all of the measurements that
% are available:
% estimate of Xk before we process the measurement at time k
% ? k l k + N = E [ x k l Y l i Y 2 , ' . . , Y k , " ' , Y k + N ] = smoothed estimate (5.6)
% where N is some positive integer whose value depends on the specific problem that
% is being solved. If we want to find the best prediction of x k more than one time
% step ahead of the available measurements, then we can form a predicted estimate.
% One way to form the predicted state estimate is to compute the expected value of
% Xk conditioned on all of the measurements that are available:
% 2 k l k - M = E [ x k j y 1 , y 2 , - . .,y k - ~ ]= predicted estimate 


\section{Reference materials}

% https://isharifi.ir/teaching/2019/IoT/[Dan_Simon]_Optimal_State_Estimation_Kalman,_H_In(BookFi).pdf
% % https://www.researchgate.net/profile/Hamid-Nasir-2/publication/297838887_Stocking_and_stand_structure_of_Sungai_Kerang_mangroves_in_matang_peninsular_Malaysia/links/5c05e070299bf169ae3049c0/Stocking-and-stand-structure-of-Sungai-Kerang-mangroves-in-matang-peninsular-Malaysia.pdf#[0,{%22name%22:%22XYZ%22},0,601,0]

(Controllability. Observability)
% https://www.ece.rutgers.edu/~gajic/psfiles/chap5.pdf

\section{Linearization}
% \documentclass{article}

% % Packages for formatting
% \usepackage[utf8]{inputenc}
% \usepackage[T1]{fontenc}
% \usepackage{geometry}
% \usepackage{titling}
% \usepackage{enumitem}
% \usepackage{amsmath}
% \usepackage{amsfonts}
% \usepackage{amssymb}
% \usepackage{graphicx}
% \usepackage{hyperref}
% \usepackage{fancyhdr}
% \usepackage{lipsum} % For generating dummy text, you can remove this in your final document
% \usepackage{amsmath}
% \usepackage{xcolor}
% \usepackage{tcolorbox}
% % \documentclass{article}

% % Define page geometry
% \geometry{a4paper, margin=1in}

% % Define title
% \title{Study Material Template}
% \author{Your Name}
% \date{\today}

% % Define header and footer
% \pagestyle{fancy}
% \fancyhf{}
% \rhead{\thepage}
% \lhead{\theauthor}
% \chead{\thetitle}
% \renewcommand{\headrulewidth}{1pt}
% \renewcommand{\footrulewidth}{0pt}

% \begin{document}

% % Title
% \maketitle

% % Table of contents
% \tableofcontents
% \newpage

% % Start of content

% \section{Introduction}
% % Your introduction here

% \section{Topic 1}
% \subsection{Subtopic 1.1}
% % Your content here

% \subsection{Subtopic 1.2}
% % Your content here

% \section{LQR}

% \subsection{INtuition behind poles}
% In control theory, the system state-space equation

% ˙x=Ax+Bu

% y=Cx+Du

% has the transfer function


% % % Define tcolorbox styles
% % \tcbset{
% %     commonstyle/.style={
% %         boxrule=0.5mm,
% %         colback=white,
% %         arc=2mm,
% %         fonttitle=\bfseries
% %     },
% %     theoremstyle/.style={
% %         commonstyle,
% %         colframe=theoremcolor
% %     },
% %     definitionstyle/.style={
% %         commonstyle,
% %         colframe=definitioncolor
% %     }
% % }


% % \begin{tcolorbox}[theoremstyle,title=Theorem]
% % The transfer function of a system is given by
% % \[ G(s) = C(sI-A)^{-1}B + D \]
% % where \( (sI-A)^{-1} = \text{adj}(sI-A) \det(sI-A) \), and the poles of \( G(s) \) are the values of \( s \) that satisfy \( \det(sI-A) = 0 \). This is exactly the characteristic equation of matrix \( A \), whose solutions are the eigenvalues of \( A \).
% % \end{tcolorbox}


% % G(s)=C(sI−A)−1B+D.

% % Since (sI−A)−1=adj(sI−A)det(sI−A), where adj(sI−A) is the adjugate of sI−A, the poles of G(s) are the numbers that satisfy det(sI−A)=0. This is exactly the characteristic equation of matrix A, whose solutions are the eigenvalues of A

% % .

% \subsection{LQR}
% In pole-placement method, we want to place the poles in the specific spots (or, we choose specific eigenvalues). But it is not intuitive where to place them, especially for complex systems, systems with numerous actuators.
% So, the new method is proposed. The key concept of the method lies in optimization of choosing K. 

% In LQR we find an optimal $K$ by choosing parameters that are important to us, specifically how well the system performs and how much effort it takes to reach this performance.

% If $Q>>R$, then we are turning the problem of 
% Let J be an additive cost function:
% \[J(x_0, p(x, t)) = \int_{0}^{\inf} g(x, u)\]
% Q - how bad if x is not where it is supposed to be.
% Q- nonnegative, poisitive semidefinite.

% if the system is a postions, velocity, and Q = [1, 0, 0, 0], [0, 1, 0, 0], [0, 0, 10, 0], [0, 0, 0, 100] we penalizez for 

% Suppose there is the best control law:
% \[u = -kx\]

% that miimizes the quadratic cost function.

% \[J = int x^T Q x + u^T R u\]
% I\textbf{Hamiltonian-Jacobi-Bellman (HJB)}

% \[min u [g(x, u) + \frac{dJ}{dx f(x, u)}] = 0\]

% Cost on effectiveness and energy to reach this effectiveness.
% \subsection{Subtopic 2.1}
% % Your content here

% \subsection{Subtopic 2.2}
% % Your content here

% % Add more sections and subsections as needed

% % End of content

% \section{Controllability. Observability}

% % \textbf{Controllability} 
% % A system is controllable on $t_0\leqt t\leqt_f$  if it possible to find control input  $u(t)$ that would drive the system to a desired state $x(t_f)$ from nay initial state $x_0$

% % \textbf{Observability}
% % A system is observable on $t_0\leqt t\leqt_f$ if it is possible to estimate exactly the state of the system $x(t_f)$, given any intial estimation error.

% % \textbf{Observability, alternative}
% % A system is observable on $t_0\leqt t\leqt_f$  if any initial state $x_0$ is uniquely determined by output $y(t)$ on that interval.

% \noindent\fbox{%
% \begin{minipage}{\textwidth}
% \begin{tcolorbox}[controllability,title=Controllability]
% A system is controllable on \(t_0 \leq t \leq t_f\) if it is possible to find a control input \(u(t)\) that would drive the system to a desired state \(x(t_f)\) from any initial state \(x_0\).
% \end{tcolorbox}

% \begin{tcolorbox}[observability,title=Observability]
% A system is observable on \(t_0 \leq t \leq t_f\) if it is possible to exactly estimate the state of the system \(x(t_f)\), given any initial estimation error.
% \end{tcolorbox}

% \begin{tcolorbox}[alternative,title=Observability (Alternative)]
% A system is observable on \(t_0 \leq t \leq t_f\) if any initial state \(x_0\) is uniquely determined by the output \(y(t)\) on that interval.
% \end{tcolorbox}
% \end{minipage}%
% }



% \end{document}


\documentclass{article}

% Packages for formatting
\usepackage[utf8]{inputenc}
\usepackage[T1]{fontenc}
\usepackage{geometry}
\usepackage{titling}
\usepackage{enumitem}
\usepackage{amsmath}
\usepackage{amsfonts}
\usepackage{amssymb}
\usepackage{graphicx}
\usepackage{hyperref}
\usepackage{fancyhdr}
\usepackage{lipsum} % For generating dummy text, you can remove this in your final document
\usepackage{xcolor}
\usepackage{tcolorbox}

% Define page geometry
\geometry{a4paper, margin=1in}

% Define title
\title{Study Material Template}
\author{Your Name}
\date{\today}

% Define header and footer
\pagestyle{fancy}
\fancyhf{}
\rhead{\thepage}
\lhead{\theauthor}
\chead{\thetitle}
\renewcommand{\headrulewidth}{1pt}
\renewcommand{\footrulewidth}{0pt}

% Define colors
\definecolor{controllable}{RGB}{50, 132, 191}
\definecolor{observable}{RGB}{65, 168, 95}
\definecolor{observalternative}{RGB}{191, 97, 106}

% Define tcolorbox styles
\tcbset{
    commonstyle/.style={
        boxrule=0.5mm,
        colback=white,
        arc=2mm,
        fonttitle=\bfseries
    },
    controllability/.style={
        commonstyle,
        colframe=controllable
    },
    observability/.style={
        commonstyle,
        colframe=observable
    },
    alternative/.style={
        commonstyle,
        colframe=observalternative
    }
}

\begin{document}

% Title
\maketitle

% Table of contents
\tableofcontents
\newpage

% Start of content

\section{Introduction}
% Your introduction here

\section{Topic 1}
\subsection{Subtopic 1.1}
% Your content here

\subsection{Subtopic 1.2}
% Your content here

\section{LQR}

\subsection{INtuition behind poles}
In control theory, the system state-space equation

\[\dot{x} = Ax+Bu\]

\[y=Cx+Du\]

has the transfer function

\[G(s)=C(sI-A)^{-1}B+D.\]

Since \((sI-A)^{-1}=\text{adj}(sI-A)\det(sI-A)\), where \(\text{adj}(sI-A)\) is the adjugate of \(sI-A\), the poles of \(G(s)\) are the numbers that satisfy \(\det(sI-A)=0\). This is exactly the characteristic equation of matrix \(A\), whose solutions are the eigenvalues of \(A\).

\subsection{LQR}
In pole-placement method, we want to place the poles in the specific spots (or, we choose specific eigenvalues). But it is not intuitive where to place them, especially for complex systems, systems with numerous actuators.
So, the new method is proposed. The key concept of the method lies in optimization of choosing \(K\). 

In LQR we find an optimal \(K\) by choosing parameters that are important to us, specifically how well the system performs and how much effort it takes to reach this performance.

If \(Q>>R\), then we are turning the problem of 
Let \(J\) be an additive cost function:
\[J(x_0, p(x, t)) = \int_{0}^{\infty} g(x, u)\]
\(Q\) - how bad if \(x\) is not where it is supposed to be.
\(Q\) - nonnegative, positive semidefinite.

if the system is a positions, velocity, and \(Q = \begin{bmatrix} 1 & 0 & 0 & 0 \\ 0 & 1 & 0 & 0 \\ 0 & 0 & 10 & 0 \\ 0 & 0 & 0 & 100 \end{bmatrix}\) we penalize for 

Suppose there is the best control law:
\[u = -kx\]

that minimizes the quadratic cost function.

\[J = \int x^T Q x + u^T R u\]

\textbf{Hamiltonian-Jacobi-Bellman (HJB)}

\[\min_u [g(x, u) + \frac{dJ}{dx f(x, u)}] = 0\]

Cost on effectiveness and energy to reach this effectiveness.

\subsection{Subtopic 2.1}
% Your content here

\subsection{Subtopic 2.2}
% Your content here

% Add more sections and subsections as needed

% End of content

\section{Taylor series approximation}


\end{document}


\section{Lyapunov theory}
\section{Lyapunov theory}

Stability is not defined only for linear systems, but for nonlinear systems as well.

There exist a small neighbourhood. Every solution that starts from this neighbourhood will asymptotically approach the same solution. 

\textbf{Asymptotic stability criteria:}\\
Autonomous dynamic system $\dot x = f(x)$ is asymptotically stable if there exists a scalar function $V = V(x) > 0$. 

% Autonomous - only \frac{\partial epen}{\partial s}on its state, does not depend on any external input. 

Note that computing $\dot V(x)$ comes along with computing $\dot x$. 
We can think of $V(x)$ as energy (pseudo-energy).

Asymptotic stability means that the system converges, and marginal stability means that the system does not diverge. 


\begin{tcolorbox}[colback=blue!10,colframe=blue!50!black,title=\textbf{Asymptotic stability}]
The system is called asymptotically stable if there exists a function $V > 0$ such that $\dot{V} < 0$, except for $V = 0, \dot{V} = 0$. 
\end{tcolorbox}

\begin{tcolorbox}[colback=green!10,colframe=green!50!black,title=\textbf{Marginal stability}]
The system $\dot{x} = f(x)$ is stable in the sense of Lyapunov if there exists $V > 0$ such that $\dot{V} \leq 0$. 
\end{tcolorbox}

\begin{tcolorbox}[colback=red!10,colframe=red!50!black,title=\textbf{Lyapunov function}]
A function $V > 0$ in this case is called a Lyapunov function. 
\end{tcolorbox}



\textbf{Example 1}
Consider the following system:
\[\dot{x} = -x\]

Let's introduce a function $V = x^2 > 0$ for all $x \neq 0$. $\dot{V} = 2x \cdot \dot{x} = 2x (-x) = -2x^2 < 0$. 
$V$ satisfies the Lyapunov criteria, so the system is (asymptotically) stable. 

\textbf{Example 2}
Consider the equation of the pendulum:
\[\ddot{q} = -\dot{q} - \sin(q)\]

% Let's introduce a function $V = q^2$.

\subsection{LaSalle's Principle}
The system is called stable if 


LaSalle's principle allows us to prove asymptotic stability even for $\dot{V}(x) \leq 0$ only for the trivial solution.


\subsection{Linear case}
Lyapunov theory applies for both nonlinear and linear systems. 
For linear systems the following feature exist:

For a system $\dot x = Ax$, we can always choose a Lyapunov candidate function in the form:

\[V = x^TSx, \] where S is positive-definite
\[\dot V = \dot x^T S x + x^T S \dot x = (Ax)^T Sx + x^TSAx = x^TA^TSx + x^TSAx = x^T (A^TS+SA) x\]




\end{document}
