\documentclass{article}

% Packages for formatting
\usepackage[utf8]{inputenc}
\usepackage[T1]{fontenc}
\usepackage{geometry}
\usepackage{titling}
\usepackage{enumitem}
\usepackage{amsmath}
\usepackage{amsfonts}
\usepackage{amssymb}
\usepackage{graphicx}
\usepackage{hyperref}
\usepackage{fancyhdr}
\usepackage{lipsum} % For generating dummy text, you can remove this in your final document
\usepackage{xcolor}
\usepackage{tcolorbox}

% Define page geometry
\geometry{a4paper, margin=1in}

% Define title
\title{Study Material Template}
\author{Your Name}
\date{\today}

% Define header and footer
\pagestyle{fancy}
\fancyhf{}
\rhead{\thepage}
\lhead{\theauthor}
\chead{\thetitle}
\renewcommand{\headrulewidth}{1pt}
\renewcommand{\footrulewidth}{0pt}

% Define colors
\definecolor{controllable}{RGB}{50, 132, 191}
\definecolor{observable}{RGB}{65, 168, 95}
\definecolor{observalternative}{RGB}{191, 97, 106}

% Define tcolorbox styles
\tcbset{
    commonstyle/.style={
        boxrule=0.5mm,
        colback=white,
        arc=2mm,
        fonttitle=\bfseries
    },
    controllability/.style={
        commonstyle,
        colframe=controllable
    },
    observability/.style={
        commonstyle,
        colframe=observable
    },
    alternative/.style={
        commonstyle,
        colframe=observalternative
    }
}

\begin{document}

% Title
\maketitle

% Table of contents
\tableofcontents
\newpage

% Start of content

\section{Maths}

\section{Introduction}
% Your introduction here

\section{Topic 1}
\subsection{Subtopic 1.1}
% Your content here

\subsection{Subtopic 1.2}
% Your content here

\section{LQR}

\subsection{Intuition behind poles}
In control theory, the system state-space equation

\[\dot{x} = Ax+Bu\]

\[y=Cx+Du\]

has the transfer function

\[G(s)=C(sI-A)^{-1}B+D.\]

Since \((sI-A)^{-1}=\text{adj}(sI-A)\det(sI-A)\), where \(\text{adj}(sI-A)\) is the adjugate of \(sI-A\), the poles of \(G(s)\) are the numbers that satisfy \(\det(sI-A)=0\). This is exactly the characteristic equation of matrix \(A\), whose solutions are the eigenvalues of \(A\).

\subsection{LQR}
In pole-placement method, we want to place the poles in the specific spots (or, we choose specific eigenvalues). But it is not intuitive where to place them, especially for complex systems, systems with numerous actuators.
So, the new method is proposed. The key concept of the method lies in optimization of choosing \(K\). 

In LQR we find an optimal \(K\) by choosing parameters that are important to us, specifically how well the system performs and how much effort it takes to reach this performance.

If \(Q>>R\), then we are turning the problem of 
Let \(J\) be an additive cost function:
\[J(x_0, p(x, t)) = \int_{0}^{\infty} g(x, u)\]
\(Q\) - how bad if \(x\) is not where it is supposed to be.
\(Q\) - nonnegative, positive semidefinite.

if the system is a positions, velocity, and \(Q = \begin{bmatrix} 1 & 0 & 0 & 0 \\ 0 & 1 & 0 & 0 \\ 0 & 0 & 10 & 0 \\ 0 & 0 & 0 & 100 \end{bmatrix}\) we penalize for 

Suppose there is the best control law:
\[u = -kx\]

that minimizes the quadratic cost function.

\[J = \int x^T Q x + u^T R u\]

\textbf{Hamiltonian-Jacobi-Bellman (HJB)}

\[\min_u [g(x, u) + \frac{dJ}{dx f(x, u)}] = 0\]

Cost on effectiveness and energy to reach this effectiveness.

\subsection{Subtopic 2.1}
% Your content here

\subsection{Subtopic 2.2}
% Your content here

% Add more sections and subsections as needed

% End of content
\section{Observer}
When the full state feedback is unavailablle, we introducce an observer to \underline{estimate} the state $x$.

% https://courses.engr.illinois.edu/ece486/fa2018/handbook/lec22.html



We created a system with a controller, but how to check the current state of the system.

We can try to estimate it with the measurements, for example form sensors. But in real life 
the task is not that trivial due to some problems:

\begin{enumerate}
    \item Lack of sensors. For a quadrotor we can not measure the height straight forwardly 
    \item Measurements can be imprecise or biased
    \item Measurements can be only made in discrete time
\end{enumerate}

The key problem arises when the output of the system $y$ is not the whole state $x$, but $y = Cx$, which
means that we are able to get the state partially.

These are just several problems that create the difficulties for us to measure the state of the system. 


Let's introduce a new idea of how to introduce an observer in our system. 

% \begin{cases}
%     \dot x = Ax + Bu
%     y = C \cdot x
%     \hat x(t) = estimate  (y(t))
%     u = -K \hat x
% \end{cases}

We estimate x $\hat x$ based on the history of y values. 


$x$ and $y$ are the state and outpuut (actual or true)

\textbf{Estimation error}

State estimation error:
\[\epsilon = \hat x - x\]

But since we do not know x, we cannot compute this error $\epsilon$.

However, we can always find $~ y = C \hat x - y$. 

Let's suggest that the dynamics should also hold for the observed state:
\[\hat \dot x = A \hat x + Bu\]

Let's introduce in our equation a linear correction law $-L ~y$. 
Since $~y = C \hat x - y$, we get:

\begin{equation}
    \hat \dot x = A \hat x + Bu + L(y - C \hat x) 
\end{equation}

This is called Luenberger observer.  

But how to find suitable observer gain $L$?

Let's subtract $\dot x = Ax + Bu$ from (?). The equation we got is the observer error dynamics:

% \begin{equation}
%     \hat \dot x - \dot x = A \hat x - Ax + L(y - C \hat x)
% \begin{equation}


% \begin{equation}
    % \dot \epsilon = (A-LC) \epsilon
% \end{equation}


With no knowledge of $x$ nad $\hat x$, we can define the stability of the system. 

The observer $\dot \epsilon = (A - LC) \epsilon$ is stable. 

% \dot x = Ax 


\subsection{Luenberger Observer}
Based on the output y, we wish to estimate input x.


\section{Filters}
In literature, the system is frequently called a Plant, which in our case is a robot. 

\underline{Static system} 

Static systemns are linear systems for whhich the relation between input and output is constant

\underline{Dynamic system} 

Dynamic systemns are linear systems for whhich the relation between input and output depends on time. 

In Laplace domain,  plant 
$Y(S) = $

Time does not appear in this equaiton, but if we rewrite it into ode form,

\[\ddot y + 5\dot y + y = u\]

For these systems, the output depends not only on the current (time-wise) value of input, but on the entire history of input values.


Static systems form algebraic linear equaitons. Dynamical systems create linear differential equaitons


For the system we can create the controller and Luenbberger observer:


\[\]

\section{Controllability. Observability}
\include{controllability_observability.tex}




\section{Kalman Filter}
CHECK! 

Without a Kalman filter, if your system is not observable, you won't be able to retrieve all the state variables xx from the output yy alone.
By using a Kalman filter, you can estimate all the state variables xx even if they are not all directly observable in yy. The filter uses the 
model of the system and the measurements to infer the values of the unobserved states.

% The way we pick our initial state estimate does not have a bias.

% Assume you could pick your initial state estimate $\hat x_0$ such that
% the initial state estimation error behaves as a random 
% variable sampled from a Gaussian distribution $x_0 ~ N(0, P_0)$

% Knowing mean 

% \[\]

% E[w_i] = 0

% All subsequent means will be $E[]$



% Let's compute the autocovariance $P_{i+1}$ knowing $P_i$ 

% \[P_{i+1} = E[] = E[(Ax)]\]


% We can assume thta the random process is uncorrelated with x, so 

% Since the two processes are uncorreleted then their covariance is 0.  

% \[P_{i+1} = E[A x_i x_i^T A^T + w_i w_i^T] \]

% Note that x_i^T A^T = P_i - a covariance of the previous step. 
% \[P_{i+1} = A P_i A^T + Q\]


% $v_i$ - the random noise sampled from thhe Gaussian distribution, which represents the sensor error. 
% \[v_i ~ N(0, R)\]


% $\hat x_{i+1}^-$ is the estimation before measurements - apriori estimate.

% \[x_i+1 \]

% $L_i$ is a control gain. 



% Apriori error \[x_{i+1}^- = x_{i+1} - \\hat x_{i+1}^-\]


% We claculate aposteriori covariance knowing our apriori covariance.


% The question is how to minimize those covariances.





% f we have all of the measurements
% up to and including time k available for use in our estimate of X k , then we can form
% an a posteriori estimate, which we denote as 2:. The "+" superscript denotes that
% the estimate is a posteriori. One way to form the a posteriori state estimate is to
% compute the expected value of x k conditioned on all of the measurements up to
% and including time k:
% 2; = E [ X k / y l ,y 2 , . . ., Y k ] = a posteriori estimate




% DERIVATIONOF THE DISCRETE-TIME KALMAN FILTER 125
% the estimate is a posteriori. One way to form the a posteriori state estimate is to
% compute the expected value of x k conditioned on all of the measurements up to
% and including time k:
% 2; = E [ X k / y l ,y 2 , . . ., Y k ] = a posteriori estimate (5.3)
% If we have all of the measurements before (but not including) time k available for
% use in our estimate of X k , then we can form an a praori estimate, which we denote
% as 2 ; . The "-"superscript denotes that the estimate is a priori. One way to form
% the a priori state estimate is to compute the expected value of 51, conditioned on
% all of the measurements before (but not including) time k:
% 2; = E [ X k l y l , y 2 , . ., Y k - l ] = a priori estimate (5.4)
% It is important to note that 2; and 2; are both estimates of the same quantity; they
% are both estimates of X k . However, 2 i is our estimate of Xk before the measurement
% Y k is taken into account, and 2: is our estimate of 21, after the measurement y k
% is taken into account. We naturally expect 2; to be a better estimate than 2 i ,
% because we use more information to compute 2;:
% 2 i =
% 2' k = estimate of Xk after we process the measurement at time k (5.5)
% If we have measurements after time k available for use in our estimate of X k , then
% we can form a smoothed estimate. One way to form the smoothed state estimate is
% to compute the expected value of x k conditioned on all of the measurements that
% are available:
% estimate of Xk before we process the measurement at time k
% ? k l k + N = E [ x k l Y l i Y 2 , ' . . , Y k , " ' , Y k + N ] = smoothed estimate (5.6)
% where N is some positive integer whose value depends on the specific problem that
% is being solved. If we want to find the best prediction of x k more than one time
% step ahead of the available measurements, then we can form a predicted estimate.
% One way to form the predicted state estimate is to compute the expected value of
% Xk conditioned on all of the measurements that are available:
% 2 k l k - M = E [ x k j y 1 , y 2 , - . .,y k - ~ ]= predicted estimate 


\section{Reference materials}

% https://isharifi.ir/teaching/2019/IoT/[Dan_Simon]_Optimal_State_Estimation_Kalman,_H_In(BookFi).pdf
% % https://www.researchgate.net/profile/Hamid-Nasir-2/publication/297838887_Stocking_and_stand_structure_of_Sungai_Kerang_mangroves_in_matang_peninsular_Malaysia/links/5c05e070299bf169ae3049c0/Stocking-and-stand-structure-of-Sungai-Kerang-mangroves-in-matang-peninsular-Malaysia.pdf#[0,{%22name%22:%22XYZ%22},0,601,0]

(Controllability. Observability)
% https://www.ece.rutgers.edu/~gajic/psfiles/chap5.pdf

\section{Linearization}
Linear models occur only in specific cases: DC motors. mass-spring damper, 3D-printer.
Most real-world systems are non-linear. 

\textit{\textbf{Taylor expansion around node.}}

Let's consider a non-linear dynamical system $\dot x = f(x, u)$ and a trajectory $x_0 = f(x_0, u_0)$

Node - any position at which robot remains static. 

Physical meaning: node is any position where the robot remains static.

The main "magic" aboout the node happens with $u_0$, which is the control that enables the robot to be static. 
\[A = \dfrac{\partial f}{\partial x}\]



\textit{\textbf{Taylor expansion along a trajectory.}}

Consider a non-linear dynamical system:
$\dot  x = f(x, u)$ and a trajectory

\[f(x, u) \approx f(x_0, u_0) + \dfrac{\partial f}{\partial x} (x - x_0) + \dfrac{\partial f}{\partial f} (u - u_0) + \text{H.O.T}\]

Since $\dot e = \dot \xi - \dot x_0$, we re-werite:

\[\dot e = Ae + Bv + \text{H.O.T}\]

Let's drop the high-order term and obtain linearization. 

\[\dot e = Ae + Bv\]

Expansion around the Node and expansion around trajectory stay the same, nothing actually changes. 



If we drop the higher-order terms from the Taylor expansion, we obtain linearization of the system dynamics. 

\[\dot e = Ae + Bv\]

Nothing changes between expansion around a node and expansion along a trajectory. 
The original function and the local approximation behave in the same way in the region.
However, the change of variables is different

So, can we linearaze around each and any point?

1) If it is a node - yes. 
2) If we want to linearize around some other point - yes, but the chnage of variables would entail slightly 
different results. 

While we had:
\[ e = x - x_0, \dot e = \dot x\], now
$\dot e$ is the differrence between the derivative of the state of our non-linear system and derivative of the state of our trajectory distance from the trajectory linearization.  

The meaning of the variables changes slightly, but the mathematical expression is the exact same expression. 


\textbf{Affine expansion}

The other way to obtain linearization, without change of varibales is as follows. 

\[f(x, u) \approx f(x_0, u_0) + A(x-x_0) + B(u-u_0) \]

Denoting \[f(x_0, u_0) - Ax_0 - Bu_0 = c\]

\[\dot x = Ax+ Bu+ c\]
c makes it not a linear model, but affine, which means it has a constant term.

It can be a constant in the case of a node, \omegar a function of time in the case of expansion along the trajectory.


Often we choose u to compensate c, so u will also be affine in this case. 


Manipulator equations (describe ???)


Cars, underwater robots are desccribed by the other equations - Euler or Lagrangian equaitons. 

Consider  Manipulator equation - quadratic, symmetric, another form of kinetic energy. 
\[ \]

H - generalized inertia matrix 
\[x = \vec{q - q_0}\]

\section{Lyapunov theory}
Stability is not defined only for linear systems, but forr nonlinear systems as well.

There exist a small neighbourhood. Every solution that starts from this neighbourhood will asymptotically approach the same solution. 

\textbf{Asymptotic stability criteria:}
Autonomous dynamic system $\dot x = f(x)$ is asymptotically stable if there exists a scalar function $V = V(x) > 0$. 

Autonomous - only \frac{\partial epen}{\partial s}on its state, does not depend on any external input. 

Note that computing $\dot V(x)$ comes along with computing $\dot x$. 
We can think of $V(x)$ as energy (pseudo-energy).

Asymptotic stability means that the system converges, and marginal stability means that the system does not diverge. 

\end{document}
