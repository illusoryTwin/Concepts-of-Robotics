While linear models are present in the real world (examples include DC motors, 3D printers, and mass-spring-damper systems), the majority of real-world systems are nonlinear (such as cars, airplanes, robot arms, underwater robots, and almost any mechanical system with linkages).

However, there is a method to describe these systems using our toolkits - through linearization.

\begin{equation*}
    \boxed{\textbf{Linearization}} \text{ is the technique that enables us to create a linear model for a nonlinear system.}
\end{equation*}

So, we can introduce linear models for nonlinear systems that could describe the system pretty well locally. We can do control for this local approximation. 

\subsection{Taylor expansion}

\begin{center}
    \textit{\textbf{Taylor expansion around node.}}
\end{center}

Let's consider a non-linear dynamical system $\dot{x} = f(x, u)$ and a Taylor expansion around the node $(x_0, u_0)$, i.e., 
$f(x_0, u_0) = 0$, $x_0 = \text{const}$, $u_0 = \text{const}$. 

\underline{Recall} the physical meaning of a \textbf{node}: it is any position at which the robot remains static. 

\[
    f(x, u) \approx f(x_0, u_0) + \frac{\partial f}{\partial x} (x-x_0) + \frac{\partial f}{\partial u} (u-u_0) + \text{HOT}
\]

HOT = high-order terms. \\

$(x_0, u_0)$ are an expansion point.

Now let's introduce new variables $e$ and $v$ that represent the distances from the expansion point:

\[
e = x - x_0
\]
\[
\dot{e} = \dot{x}
\]
\[
v = u - u_0
\]

Then we can rewrite the Taylor expansion:
\[
    \dot e \approx \frac{\partial f}{\partial x} e + \frac{\partial f}{\partial u} v + \text{HOT}
\]

\[
    \dot e \approx A e + B v + \text{HOT}
\]


Let's drop the high-order term and get an approximation. And this approximation is called \underline{linearization}. 


\[\dot e = Ae + Bv\]

In this context, $(x_0, u_0)$ is a linearization point. 

This type of linearization is proved to provide the best possible linear model for the given system.

\begin{center}
    \textit{\textbf{Taylor expansion along a trajectory.}}
\end{center}

Now let's consider a non-linear dynamical system:
\[\dot x = f(x, u)\] and a trajectory $x_0 = f(x_0, u_0)$. Note that now $x_0$ is not a point - it is a whole trajectory


\[
    f(x, u) \approx f(x_0, u_0) + \frac{\partial f}{\partial x} (x-x_0) + \frac{\partial f}{\partial u} (u-u_0) + \text{HOT}
\]

We can denote $\dot e = \dot x - \dot x_0 = f(x, u) - f(x_0, u_0)$. 

The rest stays the same: $e = x - x_0, v = u - u_0$

\[
    \dot e \approx A e + B v + \text{HOT}
\]

Let's drop HOT and obtain linearization:

\[
    \dot e = A e + B v 
\]



Basically, nothing changes between expansion around a node and expansion along a trajectory as long as we introduce the change of variables described above. 
The change of variables is slightly different, but original function and the local approximation behave in the same way in the region of the linearization point.


So, can we linearize around each and any point?

1) If it is a node - yes. 

2) If we want to linearize around some other point - yes, but the chnage of variables would entail slightly 
different results. 

While we had:
\[ e = x - x_0, \dot e = \dot x,\] now
$\dot e$ is the differrence between the derivative of the state of our non-linear system and derivative of the state of our trajectory distance from the trajectory linearization.  

The meaning of the variables changes slightly, but the mathematical expression is the exact same expression. 


\begin{center}
    \textbf{\textit{Affine expansion}}
\end{center}

The other way to obtain linearization, without change of varibales is as follows. 

\[f(x, u) \approx f(x_0, u_0) + A(x-x_0) + B(u-u_0) \]

Denoting \[f(x_0, u_0) - Ax_0 - Bu_0 = c\]

\[\dot x = Ax+ Bu+ c\]

c makes it not a linear model, but affine, which means it has a constant term.

It can be a constant in the case of a node, and a function of time in the case of expansion along the trajectory.


Often we choose u to compensate c, so u will also be affine in this case. 

\subsection{Manipulator equations}

Frequently, in robotics we deal with the following equation when describing a system:

\[H \ddot q + C \dot q  + g = \tau \] which is called a \textbf{manipulator equation}


However, cars, underwater robots are usually described by the other equations - Euler or Lagrangian equaitons. 

Let's derive linearization for the manipulator equation. 

Begin by proposing new variables: 

\[x = \vec{q - q_0}, \vec{\dot q - \dot q_0}\]
\[u = \tau - \tau_0\]

Points around which we linearize are denoted from equation:
\[\]

% \[\] 

% We can express $\ddot q$ as long as H is invertible.
% For any mechanical system, H happens to be always invertible. 


Let's introduce 

\[\phi(\dot q, q, t) = H^{-1} (\tau - C \dot q - g)\]

% % mechanical systems are second-order systems, while the state space is a 1st order system. 
% % d\dt() = 

% % Let's replace $d\dt()$ with $\dot x$. 

% % State matrices are:
% % A = []


% \[\dfrac{\partial \phi}{\partial q} = H{-1} (\dfrac{\partial C \dot q}{\partial q} -\dfrac{})\]

% \[dfrac{\partial \phi}{\partial \dot q} = \dfrac{\partial}{\partial \dot q}\]



% \subsection{Sylvester equation}

% % https://web.stanford.edu/class/ee363/lectures/inv-sub.pdf