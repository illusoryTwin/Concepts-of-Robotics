Linear models occur only in specific cases: DC motors. mass-spring damper, 3D-printer.
Most real-world systems are non-linear. 

\textit{\textbf{Taylor expansion around node.}}

Let's consider a non-linear dynamical system $\dot x = f(x, u)$ and a trajectory $x_0 = f(x_0, u_0)$

Node - any position at which robot remains static. 

Physical meaning: node is any position where the robot remains static.

The main "magic" aboout the node happens with $u_0$, which is the control that enables the robot to be static. 
\[A = \dfrac{\partial f}{\partial x}\]



\textit{\textbf{Taylor expansion along a trajectory.}}

Consider a non-linear dynamical system:
$\dot  x = f(x, u)$ and a trajectory

\[f(x, u) \approx f(x_0, u_0) + \dfrac{\partial f}{\partial x} (x - x_0) + \dfrac{\partial f}{\partial f} (u - u_0) + \text{H.O.T}\]

Since $\dot e = \dot \xi - \dot x_0$, we re-werite:

\[\dot e = Ae + Bv + \text{H.O.T}\]

Let's drop the high-order term and obtain linearization. 

\[\dot e = Ae + Bv\]

Expansion around the Node and expansion around trajectory stay the same, nothing actually changes. 



If we drop the higher-order terms from the Taylor expansion, we obtain linearization of the system dynamics. 

\[\dot e = Ae + Bv\]

Nothing changes between expansion around a node and expansion along a trajectory. 
The original function and the local approximation behave in the same way in the region.
However, the change of variables is different

So, can we linearaze around each and any point?

1) If it is a node - yes. 
2) If we want to linearize around some other point - yes, but the chnage of variables would entail slightly 
different results. 

While we had:
\[ e = x - x_0, \dot e = \dot x\], now
$\dot e$ is the differrence between the derivative of the state of our non-linear system and derivative of the state of our trajectory distance from the trajectory linearization.  

The meaning of the variables changes slightly, but the mathematical expression is the exact same expression. 


\textbf{Affine expansion}

The other way to obtain linearization, without change of varibales is as follows. 

\[f(x, u) \approx f(x_0, u_0) + A(x-x_0) + B(u-u_0) \]

Denoting \[f(x_0, u_0) - Ax_0 - Bu_0 = c\]

\[\dot x = Ax+ Bu+ c\]
c makes it not a linear model, but affine, which means it has a constant term.

It can be a constant in the case of a node, \omegar a function of time in the case of expansion along the trajectory.


Often we choose u to compensate c, so u will also be affine in this case. 


Manipulator equations (describe ???)


Cars, underwater robots are desccribed by the other equations - Euler or Lagrangian equaitons. 

Consider  Manipulator equation - quadratic, symmetric, another form of kinetic energy. 
\[ \]

H - generalized inertia matrix 
\[x = \vec{q - q_0}\]