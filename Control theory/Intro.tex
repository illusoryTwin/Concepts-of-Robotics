
\section{Introduction}
\begin{tcolorbox}[colback=green!10,colframe=green!50!black,title=\textbf{Dynamical systems}]
    Let's consider the \(n\)-th order ordinary differential equation (ODE):
    \[
    \mathbf{x}^{(n)} = \mathbf{f}(\mathbf{x}^{(n-1)}, \mathbf{x}^{(n-2)}, ..., \ddot{\mathbf{x}}, \dot{\mathbf{x}}, \mathbf{x}, t),
    \]
    where \(x(t)\) is a solution for the system, and \(t\) is an independent variable (usually time).
    This equation represents the dynamics of the system and it is called a \textbf{dynamical system}.\\
    x is called the \textbf{state} of the dynamical system.
\end{tcolorbox}


In canonical form, linear ODE is represented in the following way:
\[a_{n}z^{(n)} +a_{n-1}z^{(n-1)}+...+a_{2}\ddot z+a_{1}\dot z + a_0 z= b_0\]

The set $\{ \mathbf{x}, \ \dot{\mathbf{x}} \ ..., \ \mathbf{x}^{(n-1)} \}$ is called the \textbf{state  of the system}.

\textbf{State} of the system is a minimal set of variables that describe the system. Based on the current state and future inputs,
we can predict the behaviour of the system. \\

\subsection{Introducing input}
General form of an n-th order linear ODE with an input can be presented as follows:
%
\begin{equation}
    a_n y^{(n)} + 
    ... +
    a_2 \ddot{y} + a_1 \dot{y} + 
    a_0 y = u(t)
\end{equation}

\bigskip

State-space representation of a linear system with an input is:
%
\begin{equation}
    \dot x = Ax + Bu
\end{equation}

$A$ is called a \textbf{state matrix} and x is a \textbf{state vector}, 

$B$ is called a \textbf{control matrix} and $u$ is a \textbf{control vector}.

$u$ might be either a scalar or a vector. 

\subsection{Introducing output}
Equations might also have an output, which can have plenty of physical meanings and interpretations. Let's list some of them:
what we measure (position and orientation of a motor), what we want to control (the height of the quadrotor).

Output is usually defined as $y$. 

Example of system wth input and output:

\begin{equation}
    \begin{cases}
    \mathbf{\dot{x}}=\mathbf{A}\mathbf{x} + \mathbf{B}\mathbf{u} \\
    \mathbf{y}=\mathbf{C}\mathbf{x}
    \end{cases}
\end{equation}

If $u$ and $y$ are scalars, the system is called \emph{single-input single-output (SISO)}, if they are vectors - \emph{multi-input multi-output (MIMO)}.

\subsection*{Linear systems}
In case if relationships between state, output and control are linear, we can model system in following form:
\begin{equation}
\begin{cases}
\mathbf{\dot{x}} =\mathbf{A}\mathbf{x} + \mathbf{B}\mathbf{u} \\
\mathbf{y}=\mathbf{C}\mathbf{x} + \mathbf{D}\mathbf{u}
\end{cases}
\end{equation}


Where
\begin{itemize}
    \item $\mathbf{x} \in \mathbb{R}^n$: states of the system
    \item $\mathbf{y} \in \mathbb{R}^l$: output vector
    \item $\mathbf{u} \in \mathbb{R}^m$: control inputs
    \item $\mathbf{A} \in \mathbb{R}^{n \times n}$: state matrix
    \item $\mathbf{B} \in \mathbb{R}^{n \times m}$: input matrix
    \item $\mathbf{C} \in \mathbb{R}^{l \times n}$: output matrix
    \item $\mathbf{D} \in \mathbb{R}^{l \times m}$: feedforward matrix
\end{itemize}


If matrices A, B, C, D are time-independent, then we call such systems \textbf{time-invariant.} 

More frequently we work with systems when the output does not depend on control. 

\begin{equation}
    \begin{cases}
    \mathbf{\dot{x}}=\mathbf{A}\mathbf{x} + \mathbf{B}\mathbf{u} \\
    \mathbf{y}=\mathbf{C}\mathbf{x}
    \end{cases}
    \end{equation}


    
\subsection{ODE to State-Space conversion}
\[
\dddot{y} + a_2 \ddot{y} + a_0 y = u
\]
\[
\begin{bmatrix}
\dot{y} \\
\ddot{y} \\
\dddot{y}
\end{bmatrix} = 
\begin{bmatrix}
0 & 1 & 0 \\
0 & 0 & 1 \\
-a_0 & -a_1 & -a_2
\end{bmatrix}
\begin{bmatrix}
y \\
\dot{y} \\
\ddot{y}
\end{bmatrix} + 
\begin{bmatrix}
0 \\
0 \\
u
\end{bmatrix}
\]

\subsection{State-Space to ODE conversion}

\[
\begin{cases}
\dot{x} = Ax \\
y = Cx
\end{cases}
\]

We need to represent this system as an ODE in the form:
\[y^{(n)} = d_{n-1}y^{(n-1)} + d_{n-2}y^{(n-2)} + \dots + d_1 \dot{y} + d_0 \]

Let's take the derivative of \(y\):
\[\dot{y} = C\dot{x} = CAx\]
\dots
\[y^{(n)} = CA^{(n)} x\]

\[y = \begin{bmatrix} C \\ CA \\ \dots \\ CA^{(n-1)} \end{bmatrix} x = Ox, \]

\(O\) is called the \textbf{observability matrix}.

\[x  = O^{-1} y \]
Then, 
\[y^{(n)} = CA^{(n)} x = CA^{(n)}O^{-1} y = CA^{(n)}O^{-1}  \begin{bmatrix} y \\ \dot{y} \\ \vdots \\ y^{(n-1)}\end{bmatrix}\]
