\section{Laplace Transform}

\subsection{Laplace Transform}

The Laplace transform is an integral transformation that converts a function of a real variable \(t\) (time domain) to a function of a complex variable \(s\) (frequency domain). 

We can think of Laplace transform ass as general case of Fourier transform (Steve Brunton).

The Laplace transform is a tool for solving differential equations by transforming them into algebraic equations.

\textbf{The Laplace transform }of a function \(f(t)\) is given as:
\begin{equation}
    F(s) = \mathcal{L} \{ f(t)\} = \int_0^\infty f(t) e^{-st}dt
\end{equation}

where \(F(s)\) is called an \textbf{image} of the function and \(s=\alpha +\beta i \) is a complex frequency.

\subsubsection{Some Useful Properties}

Linear properties:
\begin{equation}
    {\mathcal {L}}\{f(t)+g(t)\}={\mathcal {L}}\{f(t)\}+{\mathcal {L}}\{g(t)\}
\end{equation}

\begin{equation}
    {\mathcal {L}}\{af(t)\}=a{\mathcal {L}}\{f(t)\}
\end{equation}

Final value theorem:
\begin{equation}
    f(\infty )=\lim _{s\to 0}{sF(s)}
\end{equation}

The final value theorem is useful because it gives the long-term behavior for a particular function.

\subsubsection{Inverse Laplace Transform}

The inverse Laplace transform transforms the image of your function \(F(s)\) from the frequency domain to the time domain \(x(t)\):
\begin{equation}
    f(t) = {\mathcal {L}}^{-1}\{F\}(t) = {\frac {1}{2\pi i}}\lim _{T\to \infty }\int _{\gamma -iT}^{\gamma +iT}e^{st}F(s)\,ds
\end{equation}

However, in practice, we mostly use precalculated Laplace transforms and then try to decompose the image \(X(s)\) into known transforms of functions obtained from a table, and construct the inverse by inspection, or just use some symbolic routines.

\subsection{Laplace Transform of a Function's Derivative}

For us, one of the most useful properties of Laplace transform is that if we apply it to the derivative of a given variable, it will result in the following:

\begin{equation}
    \mathcal{L}\left\{\frac{dx(t)}{dt}\right\} = s \mathcal{L}\left(x\right) = s X(s)
\end{equation}

which is true for \(x(0) = 0\).

Thus, we can define a **derivative operator**:
\begin{equation}
    \frac{dx}{dt} \xrightarrow{\mathcal{L}} s X(s)
\end{equation}

The proof is as follows, using the definition of Laplace transform:
\begin{equation}
    \mathcal{L}\left\{\frac{dx}{dt}\right\} = \int_0^\infty \frac{dx}{dt} e^{-st}dt
\end{equation}

Then using integration by parts:
\begin{equation}
    \int_0^\infty \frac{dx}{dt} e^{-st}dt =  \left[x e^{-st} \right]_0^\infty - \int_0^\infty -se^{-st} x dt
\end{equation}

which yields:
\begin{equation}
    \left[x e^{-st} \right]_0^\infty + s\int_0^\infty e^{-st} x dt = x(0) + s\mathcal{L}\{x(t)\} = x(0) + sX(s)
\end{equation}

By induction, it can be shown that:
\begin{equation}
    {\mathcal {L}}\left\{\frac{d^{n}x}{dt^{n}}(t)\right\}=s^{n}\cdot {\mathcal {L}}\{x(t)\}+s^{n-1}x(0)+\cdots +x^{(n-1)}(0)
\end{equation}

\subsubsection{Applications to Linear ODEs}

Let us consider the following ODE:
\begin{equation}
    a_{n}x^{(n)} +a_{n-1}x^{(n-1)}+...+a_{2}\ddot x+a_{1}\dot x + a_0 x= u_{m}b^{(m)} +b_{m-1}u^{(m-1)}+...+b_{2}\ddot u+b_{1}\dot u + b_0 u
\end{equation}
Notice that we introduce a new variable that we call the input \(u\) (control).

Applying the inverse Laplace transform with zero initial conditions yields:
\begin{equation}
    \begin{aligned}
        &a_{n}s^{(n)}X(s) +a_{n-1}s^{(n-1)}X(s)+\cdots+a_{2} s^2 X(s)+a_{1}s X(s) + a_0 X(s) \\
        &= b_{m}s^{(m)}U(s) +b_{m-1}s^{(m-1)}U(s)+\cdots+b_{2}s^2 U(s)+b_{1}sU(s) + b_0 U(s)
    \end{aligned}
\end{equation}


\subsubsection*{Example 1}
\begin{equation}
    \dddot y + a\dot y + by = u
\end{equation}

\begin{equation}
    S^2 Y(S) + ASY(S) + BY(S) = U(S)
\end{equation}

\begin{equation}
    Y(S) = \frac{1}{(S^2 + AS + B)} U(S)
\end{equation}

This form is called a \textbf{transfer function}. 

\subsubsection*{Example 2}
\begin{equation}
    2 \dddot y + 5 \dot y - 40 y = 10 u
\end{equation}

\begin{equation}
    2SY(S) - 4Y(S) = U(S)
\end{equation}

\begin{equation}
    Y(S)(2S - 4) = U(S)
\end{equation}

\begin{equation}
    Y(S) = \frac{1}{(2S - 4)} U(S)
\end{equation}

Consider the ODE with the input \(u\).

\subsection{State-Space to Transfer Function}

Let's study the relation between the input and the output of the dynamical system. 

\[
\begin{cases}
    \dot x = Ax + Bu \\
    y = Cx + Du
\end{cases}
\] 

Let's consider the 1st equation and rewrite it in the form:
\[SI X(S) = AX(S) + BU(S)\]
\[(SI-A) X(S) = BU(S)\]
\[X(S) = (SI-A)^{-1}BU\]

Then the initial system can be rewritten in the following form:
\[
\begin{cases}
    X = (SI-A)^{-1}BU\\
    Y = C((SI-A)^{-1}B + D)U
\end{cases}
\]
